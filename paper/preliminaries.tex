% \ndhy{This section is a work in progress. It will be completed in the next few days.}
\paragraph{Notation.}
For $n\in\mathbb{N}$, let $[n]=\{1,\dots,n\}$.
Vectors are in bold, e.g., $\mathbf{a}=(a_1,\ldots,a_\ell)$.
Concatenation is $\|$.
For a (possibly randomized) algorithm $\mathsf{Alg}$, we write $y \leftarrow \mathsf{Alg}(x)$ for its output.
The security parameter is $\lambda$, and $\mathrm{negl}(\lambda)$ denotes a negligible function.
\subsection{Algebraic setting and basic primitives}
\label{subsec:algebra}
Let $\mathbb{F}$ be a prime field of order $q$ and $\mathbb{G}$ a cyclic group of order $q$ with generator $g$, where discrete logarithms are hard.
We use SHA\mbox{-}256 as $\mathsf{H}:\{0,1\}^\ast\!\rightarrow\!\{0,1\}^{256}$.
The issuer uses a digital signature scheme $\mathsf{Sig}_I=(\mathsf{KeyGen}_I,\mathsf{Sign}_I,\mathsf{Verify}_I)$ (e.g., ECDSA/RSA), and the device uses $\mathsf{Sig}_D$ for nonce-bound session signatures; both are assumed EUF\mbox{-}CMA.

\paragraph{Pedersen vector commitments.}
Fix generators $\mathbf{g}=(g_1,\ldots,g_\ell)$ and $h$ in $\mathbb{G}$.
For $\mathbf{a}\in\mathbb{F}^\ell$ and $r\in\mathbb{F}$ define
\[
\mathsf{Com}(\mathbf{a};r)=\prod_{i=1}^{\ell} g_i^{a_i}\cdot h^{r}\in\mathbb{G}.
\]
$\mathsf{Com}$ is perfectly hiding and binding under discrete\mbox{-}log hardness.
We will need efficient \emph{openings} and \emph{equality proofs} for Pedersen commitments; we realize these via inner-product arguments (IPAs) in the Hyrax framework~\cite{SP:WTSTW18}.

\paragraph{Hyrax commitments.}
Hyrax commitments \cite{SP:WTSTW18} allow us to commit to a multilinear extension $\widetilde{P}$ of a function $P:\{0,1\}^{\log n} \rightarrow \mathbb{F}$. 
Notably, we can express $\widetilde{P}(x_1, \dots, x_{\log n})$ as a vector-matrix-vector product $\vec{v_L} P \vec{v_R}$, where:
\begin{itemize}
    \item $P$ is a square matrix of evaluations of $P$ on the $\{0,1\}^{\log n}$ hypercube, represented as $\langle \vec{p_i}\rangle_{i \in \sqrt{n}}$ where $\vec{p_i} \in \mathbb{F}^{\sqrt{n}}$ are columns of $P$
    \item $\vec{v_L}(x_1, \dots, x_{(\log n)/2}) = \langle \widetilde{\chi}_{(b_1, \dots, b_{(\log n)/2})}(x_1, \dots, x_{(\log n)/2})\rangle _{(b_1, \dots, b_{(\log n)/2})\in \{0,1\}^{(\log n)/2}}$, and 
    \item $\vec{v_R}(x_{(\log n)/2+1}, \dots, x_{\log n}) = \langle \widetilde{\chi}_{(b_{(\log n)/2+1}, \dots, b_{\log n})}(x_{(\log n)/2}, \dots, x_{\log n})\rangle_{(b_{(\log n)/2}, \dots, b_{\log n}) \in \{0,1\}^{(\log n)/2}}$,
\end{itemize}
and $\widetilde{\chi}_{b_1, \dots, b_m} (x_1, \dots, x_m) = \prod_{i \in [m]} \widetilde{\chi}_{b_i}(x_i)$, where
$\widetilde{\chi_{b_i}}(x_i)$ is the multilinear extension of the function $\chi_{b_i}(x_i): \{0,1\} \rightarrow \mathbb{F}$ given by $\chi_{b_i}(x_i) = x_ib_i + (1-x_i)(1-b_i)$, which equals $1$ if $x_i = b_i$ and $0$ otherwise.

Then our Hyrax commitment scheme is given by:
\begin{itemize}
    \item $\textbf{Setup}(1^{\lambda}, n) = (g_i)_{i \in [\sqrt n]}$ where $g_i \in \mathbb{G}$ are elements of the elliptic curve group $\mathbb{G}$ over which we compute our Pedersen commitments.
    \item $\textbf{Commit}_h(pp, P) \rightarrow (c,S)$, where 
    $c = \{c_i\}_{i \in [\sqrt{n}]}$ and $S = \{S_i\}_{i \in [\sqrt{n}]}$, and
    $(c_i, S_i) \leftarrow Commit_p(pp, \langle \vec{p_i} \rangle)$
    where $Commit_p$ is the Pedersen vector commitment scheme.
    \item $\textbf{Eval}_h(pp, c, r, v, n; P, S)$, where $r$ is the opening point and $v$ is the claimed evaluation of $P(r)$, is given by:
    \begin{itemize}
        \item Prover and Verifier compute $\vec{v_L} = \vec{v_L}(r_1, \dots, r_{(\log n)/2})$, $\vec{v_R} = \vec{v_R}(r_{(\log n)/2+1}, \dots, r_{\log n})$, 
    and $C = \sum_{i \in [\sqrt{n}]} \vec{v_L}_i c_i$ (where addition is over $\mathbb{G}$)
        \item Prover and Verifier engage in an interactive IPA protocol to prove that $P(r)$ is a dot product of $v_R$ and $C$ 
    \end{itemize} 
\end{itemize} 

\subsection{Zero-knowledge proving interface}
\label{subsec:zk-interface}
We use a \emph{transparent} general-purpose SNARK in the Spartan family~\cite{C:Setty20}, instantiated over $\mathbb{F}$ for R1CS instances.
Algorithms are
\[
\mathsf{pp} \leftarrow \mathsf{Setup}(1^\lambda),\quad
\pi \leftarrow \mathsf{Prove}(\mathsf{pp},x,w),\quad
b \leftarrow \mathsf{Verify}(\mathsf{pp},x,\pi)\in\{0,1\},
\]
where $x$ is public input and $w$ the witness.
(We reference succinct CRS-based systems such as Groth16~\cite{groth2016size} later only for performance comparisons.)

\paragraph{R1CS instance.}
An \textit{R1CS instance} is a tuple $(\mathbb{F}, A, B, C, io, n, m)$, 
where $io$ denotes the public input and output of the instance, 
$A, B, C \in \mathbb{F}^{n \times n}$, where $m \geq |io| + 1$
and there are at most $m$ non-zero entries in each matrix. An instance is said to be \textit{satisfiable}
if there exists a witness $\vec{w} \in F^{n - |io| - 1}$ such that
$(A \cdot Z) \circ (B \cdot Z) = (C \cdot Z),$
where $\vec{Z} = (io, 1, \vec{w})$, and $\circ$ is the Hadamard (entry-wise) product.

\paragraph{} Throughout the paper, assume that we are dealing with sparse R1CS instances, where $m = O(n)$.

\paragraph{R1CS as a language} We let the language 
\[
R_{\text{R1CS}} = \{ \langle x = (\mathbb{F}, A, B, C, io, n, m): \text{$x$ is satisfiable }\rangle \}.
\]
The language $R_{\text{R1CS}}$ is NP-complete.

\paragraph{} A Spartan zero-knowledge proof is an argument of knowledge for the R1CS language.
The Prover can prove not only that a given instance $x \in R_{\text{R1CS}}$, but knowledge of the corresponding witness $\vec{w}$.   
At a high level, this application considers R1CS instances that represent the computational structure of ownership of a valid credential, along with any other desired properties about the credential. 
Thus a valid argument of knowledge implies knowledge of the underlying credential that has the claimed properties, which is necessary for real-world authentication.

% \subsection{Relations used by \textsf{zkID}}
% \label{subsec:relations}
% Let $\mathsf{Parse}$ be a deterministic parser that extracts $(\mathbf{a},\mathbf{s},\mathbf{h},\sigma_I,\mathsf{meta})$ from a token $\tau$ (attributes, salts, digests, issuer signature, metadata), consistent with the chosen format (SD\mbox{-}JWT/mDL).

% \paragraph{Prepare relation $\mathcal{R}_{\mathrm{prep}}$.}
% \dots
% \paragraph{Show relation $\mathcal{R}_{\mathrm{show}}$.}
% \dots

\subsection{High-Level Credential Presentation Flow}\label{sec:high-level-flow}

The following outlines the high-level protocol for a credential presentation.

\begin{enumerate}
\item The Prover receives a signed credential from an Issuer to be stored securely in their wallet, issued to the Wallet Secure Cryptographic Device (WSCD) public key.
\item At presentation time, the Verifier sends over challenge $\texttt{nonce}_V$ for device-binding verification.
\item The Prover signs the challenge \texttt{nonce} with the public key $p_U$ controlled by their WSCD and specified in their credential.
\item The Prover computes two separate but linked zero-knowledge proofs $\pi_{\text{\texttt{prepare}}}, \pi_{\text{\texttt{show}}}$ which together cover the following statements: SD-JWT parsing, verification of the SD-JWT Issuer signature, 
proper disclosures and/or arbitrary predicates on the disclosures, and device-binding. 
(i.e.\ checks the \texttt{nonce} signature against their public key); then sends $\pi_{\text{\texttt{prepare}}}, \pi_{\text{\texttt{show}}}$ to the Verifier.
\item The Verifier verifies $\pi_{\text{\texttt{prepare}}}, \pi_{\text{\texttt{show}}}$ independently, and also verify that they are linked; grants Prover access to some service based on their credential disclosures.
\end{enumerate}

\subsection{Security model}
\label{subsec:security-model}

In our security model, we assume that the Prover is malicious, and that each Verifier is semi-honest, meaning that if the Prover presents a valid proof that they own a credential with some property, the Verifier will grant access to any services for which the property suffices.


For security on the Verifier's side, our soundness analysis considers the probability that a Prover without real ownership of a valid credential can generate a false proof of ownership. 


For security on the Prover's side, we guarantee that our proofs are zero-knowledge, so that a semi-honest computationally bounded Verifier cannot get any additional private information about the Prover's credential given the proof, beyond what is publically revealed in the proof.
In particular, we do not consider the case where the Verifier is malicious, where e.g. a false Verifier pretends to be an authorized Verifier. The problem of Verifier identity lies outside the scope of this paper.


Futhermore, we assume that Verifiers can collude, i.e. that Verifiers $V_1, \dots, V_N$ that have receieved proofs $\{\pi_1\}, \dots, \{\pi_N\}$ from a given Prover $P$ can compute functions $f(\pi_1, \dots, \pi_N)$. 
Therefore, we desire the \textbf{unlinkability property}: given $pi_1, \dots, \pi_N$, the Verifiers should not be able to determine whether or not any two of these proofs came from the same Prover $P$.
Note that this requires the Prover to re-randomize each presentation's proof; a static zero-knowledge proof of the same statement, while not revealing private credential information, will still look the same.
It is possible that Verifier's can effectively de-anonymize a Prover by linking their anonymous activity across presentations and analyzing corresponding metadata, e.g. time of presentation.

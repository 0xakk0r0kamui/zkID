%%%% IACR Transactions TEMPLATE %%%%
% This file shows how to use the iacrtrans class to write a paper.
% Written by Gaetan Leurent gaetan.leurent@inria.fr (2020)
% Public Domain (CC0)


%%%% 1. DOCUMENTCLASS %%%%
\documentclass{iacrtrans}
%%%% NOTES:
% - Change "journal=tosc" to "journal=tches" if needed
% - Change "submission" to "final" for final version
% - Add "spthm" for LNCS-like theorems


%%%% 2. PACKAGES %%%%
\usepackage{lipsum} % Example package -- can be removed
\usepackage{booktabs}
\usepackage{mdframed}
\usepackage{amsmath,amsfonts,amssymb}
\usepackage{geometry}
\usepackage{enumitem}
\usepackage{hyperref}
\usepackage{longtable}
\usepackage{pdflscape}

\usepackage{framed}

\setcounter{MaxMatrixCols}{30}


%%% COMMANDS
\newcommand{\jbel}[1]{{\color{blue}{}jbel: #1}}
\newcommand{\ndhy}[2]{{\color{blue}{}ndhy: #2}}
\newcommand{\rand}{\overset{{\scriptscriptstyle\$}}{\leftarrow}}

%%%% 3. AUTHOR, INSTITUTE %%%%
%\author{Jane Doe\inst{1,2} \and John Doe\inst{1}}
%\institute{
%  Institute A, City, Country, \email{jane@institute}
%  \and
%  Institute B, City, Country, \email{john@institute}
%}
%%%% NOTES:
% - We need a city name for indexation purpose, even if it is redundant
%   (eg: University of Atlantis, Atlantis, Atlantis)
% - \inst{} can be omitted if there is a single institute,
%   or exactly one institute per author

\author{The zkID Team}
\institute{ Ethereum Foundation}

%%%% 4. TITLE %%%%
\title{Technical Overview: zkID for the EUDI Wallet}
%%%% NOTES:
% - If the title is too long, or includes special macro, please
%   provide a "running title" as optional argument: \title[Short]{Long}
% - You can provide an optional subtitle with \subtitle.

\begin{document}

\maketitle


%%%% 5. KEYWORDS %%%%
\keywords{Anonymous credential \and programmable zkp}


%%%% 6. ABSTRACT %%%%
%\begin{abstract}
%  Main deliveries: 1. Technical report on zk component for the digital id wallet 2. A comparison with current works 3. Applying to EUDI.
%\end{abstract}


%%%% 7. PAPER CONTENT %%%%
% \section{Introduction}
% \label{sec:introduction}
% Introducing DI and Selective disclosure. From Selective Disclosure to VCs lead to Anonymous credential.
% In Anonymous credential, talking about related works and their main approaches.
% \textit{In this section, we will present the definition of the scheme, the need for it, and our scope of work.}
% 
It is widely recognized that digital identity systems must deliver security, perform reliably on everyday hardware, and preserve user privacy across multiple sessions, and it has also been observed that these requirements become most challenging when deployments move beyond controlled environments to heterogeneous devices, fluctuating networks, and varied implementations~\cite{PoPETS:KruPaiRujKan24}. Within the European Union’s European Digital Identity (EUDI) Architecture \& Reference Framework~\cite{EU:EUDI23}, these aims are stated explicitly through selective disclosure and unlinkability. Achieving security, reliability, and privacy in digital identity systems requires not only robust cryptographic choices but also practical integration into real-world environments. We consider a typical deployment: wallets run on commodity smartphones, verifiers operate within standard web infrastructures, and network conditions vary widely \cite{EU:EUDI23}. In such settings, stable identifiers or protocol patterns can enable unintended correlation of user interactions, compromising privacy. Effective designs must ensure predictable performance under load, minimal resource demands on devices, and low server-side verification costs, while avoiding non-standard cryptographic assumptions that hinder adoption \cite{NIST:Grassi17}. Our zkID system addresses these challenges by prioritizing unlinkability and deployability, as we explore in subsequent sections.

Security and privacy are distinct requirements in digital identity systems: while authenticity and integrity can be ensured, linkability often persists, compromising user privacy~\cite{PoPETS:KruPaiRujKan24,JC:FeiFiaSha88}. We work with the usual roles—issuer, holder, verifier—and with two flows: issuance and presentation. Correlation can arise from deterministic disclosure structures, from identifiers that remain stable (intended or incidental), and from protocol- or transport-level regularities across sessions. Unless policy authorizes correlation, two valid presentations should not be tied back to the same credential; this is the privacy target used here.

Standardized formats such as Selective Disclosure JWT (SD-JWT)~ \cite{IETF:FetYasCam25} and ISO mobile Driving Licence (mDL)~\cite{ISO:18013-5} were designed to minimize disclosed attributes and do so effectively. Even so, disclosure encodings and patterns can be stable enough for cooperating verifiers to correlate presentations over time. Pairing-based credentials (e.g., BBS+\cite{C:BCTV14}) address unlinkability at the credential layer; however, reliance on pairing-friendly curves complicates certification paths and leaves post-quantum migration unsettled in ecosystems that prioritize NIST-standardized primitives~\cite{EC:TesZhu23b}.

A third direction wraps standardized credentials in general-purpose zero-knowledge proofs. This can suppress correlating structure, yet it reintroduces a familiar fork: heavier verification without a common reference string, or a universal setup whose governance is difficult to reconcile with open, multi-party standardization. The question that follows is straightforward to state, although less simple to answer:
\begin{center}
    \textit{Can zero-knowledge be layered over existing, standardized credentials so that unlinkability holds against cooperating verifiers, while practical latency and verifier cost are preserved, and without a universal trusted setup or dependence on non-standard curves?}
\end{center}

For a solution to be acceptable, it should satisfy four requirements. First, deployability: it must interoperate with mDoc/JWT data structures and existing PKI (ECDSA or RSA) without changes to issuer processes or secure elements. Second, performance: proof generation and verification must remain within mobile and web latency budgets, and expensive steps (e.g., issuer-signature checks) should be cached or batched when possible. Third, unlinkability: distinct presentations of the same credential must be non-correlatable under the stated threat model, including colluding verifiers. Finally, transparency and post-quantum agility: no trusted setup and no pairing-friendly curves; rely on standard assumptions (e.g., discrete logarithm) and allow PQ replacements without redesigning the protocol.

\subsection{Our zkID Construction}

Motivated by these requirements, we propose \emph{zkID}, a modular zk-SNARK wrapper aimed at real-world identity workflows. At a high level, the design follows a simple rule: pre-process once, reuse later. Proof work is split into two phases:
\begin{itemize}
    \item \textbf{Prepare.} Run infrequently. The issuer’s ECDSA signature is verified, credential attributes are parsed from SD-JWT (or related formats), and SHA-256 hashes are computed over intended disclosures. The results are committed with Hyrax vector commitments and can be cached per credential~\cite{cryptoeprint:2017/1132}.
    \item \textbf{Show.} Run per presentation: selected attributes or predicates are disclosed, the device-bound nonce signature is validated, and equality is proved against cached commitments from the \emph{Prepare} phase.
\end{itemize}

Our construction leverages Spartan interactive-oracle proofs over the Tom256 elliptic curve, aligning with ECDSA infrastructures (e.g., NIST P-256) and using Hyrax commitments for efficient linking \cite{C:Setty20,cryptoeprint:2017/1132}. Our evaluation considers a standard adversary model with malicious holders, colluding verifiers, and honest-but-curious issuers, deferring quantum adversaries and side-channel attacks to future work. To highlight differences in zkID’s approach, we review related privacy-preserving credential systems, comparing their strategies for unlinkability and deployability with our solution \cite{cryptoeprint:2024/2010,cryptoeprint:2024/2013}.

\subsection{Related Work}

\paragraph{Anonymous Credentials from ECDSA.}
Matteo Frigo and Abhi Shelat (Google) propose an anonymous credential scheme for legacy-deployed ECDSA that preserves deployability on existing infrastructures (e.g., P-256, SHA-256) without issuer or device changes and without trusted setup, addressing reluctance to upgrade infrastructures that only support RSA/ECDSA~\cite{cryptoeprint:2024/2010}. Their design tackles the bottleneck of zero-knowledge proofs over non-NTT-friendly curves by building a proof stack around sum-check and the Ligero argument~\cite{CCS:AHIV17}, introducing specialized ECDSA circuits and efficient Reed–Solomon encodings, and adding a witness-consistency mechanism to keep common witness values aligned across arithmetic over $\mathbb{F}_{p}$ (ECDSA) and $\mathrm{GF}(2^k)$ (SHA-256). As reported, the system generates a proof for a single ECDSA signature in 60 ms and a zero-knowledge proof for the ISO/IEC 18013-5 mDL presentation flow in 1.2 s on mobile devices~\cite[\S5.3,\S6.2]{cryptoeprint:2024/2010}. Device binding follows the mDL live-challenge pattern. The main trade-off is comparatively larger proofs and higher verifier workload than succinct CRS-based SNARKs.

\paragraph{Crescent Credentials.}
Christian Paquin, Guru-Vamsi Policharla, and Greg Zaverucha introduce Crescent to upgrade privacy of existing credentials (JWTs, mDL) with selective disclosure and unlinkability without changing issuance processes~\cite{cryptoeprint:2024/2013}. Crescent achieves fast online presentations with compact proofs, while relying on a two-phase workflow:
\begin{itemize}
  \item Prepare (offline, once per credential): verify issuer signatures, decode and parse the credential into attributes, and produce a Pedersen vector commitment via a Groth16 proof. Reported timings are approximately 27 s for a 2 KB JWT and 140 s for an mDL, with universal-setup parameters of size approximately 661 MB–1.1 GB~\cite[\S4]{cryptoeprint:2024/2013}.
  \item Show (online, per presentation): re-randomize the Groth16 proof and attach a few $\Sigma$-proofs over committed attributes; typical latencies are approximately 22 ms for JWT and 41.2 ms for mDL, with proofs around 1 KB. Optional device binding increases these costs (e.g., about 315 ms Show, about 184 ms verify; proof size about 15 KB)~\cite[\S4]{cryptoeprint:2024/2013}.
\end{itemize}
Crescent exposes a modular committed-attribute interface that admits sub-provers for range checks, credential linking, and binding to session context or device by proving knowledge of a signature under a committed public key. The principal trade-offs are reliance on pairing-based Groth16 with a trusted setup and large universal parameters, and a comparatively heavy (but amortized) offline Prepare phase; in its current form Crescent is not post-quantum secure~\cite{groth2016size}.

\paragraph{zkID.}
zkID is a modular, transparent (no-CRS) zero-knowledge wrapper for standardized credentials, prioritizing unlinkability and deployability. We adopt the same two-phase workflow as Crescent but avoid a universal setup, and we retain ECDSA compatibility as in Google’s design while moving issuer verification to the offline phase to reduce online verifier work. Key distinctions from prior work include:
\begin{itemize}
  \item Transparency and PQ agility: no trusted setup; standard discrete-log assumptions; an architecture designed to swap in PQ vector commitments in future instantiations.
  \item Deployment compatibility: interoperates with mDL/SD-JWT and existing PKI (ECDSA or RSA) without changes to issuer processes or secure elements.
  \item Reduced online burden: relative to prior ECDSA-based designs, issuer-signature verification and parsing move to the offline Prepare phase, reducing online compute.
  \item Avoidance of pairing-based cryptography: unlike pairing-based schemes, zkID avoids pairing-friendly curves.
  \item Verifier cost: reduces verifier work relative to prior ECDSA-based designs under our constraints while preserving unlinkability under the stated threat model.
\end{itemize}

In summary, zkID shows that a modular, standards-aligned zero-knowledge wrapper over existing credential formats can be built. Compared to pairing-based schemes, it avoids dependence on pairing-friendly curves; compared to universal-setup SNARK wrappers, it avoids a trusted setup; and relative to prior ECDSA-based designs, it reduces verifier cost while preserving unlinkability under our model. The construction is transparent and PQC-agile; the current instantiation is classical and not quantum-resistant.

In our evaluation, zkID achieves practical performance on commodity hardware. For a typical credential with ..., the one-time \emph{Prepare} phase completes in ... seconds on a standard laptop (or mobile) (..., ...GB RAM), while the per-presentation \emph{Show} phase takes less than ... ms on a mid-range smartphone. Proof sizes are approximately ... KB for the full protocol, and verifier time remains below ... ms on a web server. These results demonstrate that zkID meets latency and resource targets for mobile and web deployments, while preserving unlinkability and avoiding trusted setup. Detailed benchmark results and further breakdowns will be provided in Section~\ref{sec:experiments}.

Several aspects are left out of scope. We do not provide comprehensive revocation, explicit defenses against hardware side channels, or end-to-end wallet-integration benchmarks, and a concrete instantiation of lattice-based vector commitments is deferred to future work.
% \subsection{Problems and our scope of work}
% \begin{itemize}
%     \item What are we trying to solve, which pain points? 
%     \begin{itemize}
%         \item \textbf{\href{https://mirror.xyz/privacy-scaling-explorations.eth/zRM7qQSt_igfoSxdSa0Pts9MFdAoD96DD3m43bPQJT8}{This doc} dive deeper in the current problems and some current tool could be used \\ in future solutions}
%         \item \textbf{Answer described in this docs, \href{https://www.notion.so/pse-team/External-zkID-ZKP-Wallet-Unit-Proposal-1bad57e8dd7e80c98d73fc7e7666375d?pvs=25\#1bad57e8dd7e8059a446ca7b1dc31323}{Short-term Deliverables $\&$ Further Exploration}}
%         \item \textbf{And in this \href{https://pse.dev/en/projects/zk-id}{doc}} 
%     \end{itemize}
% \end{itemize}
% \subsection{Our achievements}
% \begin{itemize}
%     \item We need to describe zkID as simple as possible. What is it based on, like, well-known terms, and well-known algorithms?... \textit{I don't know the final zkID construction yet. Based on \href{https://pse-team.notion.site/zkID-Team-Strategy-Proposal-db3c5788dc7a4916a33e580a79177053}{\textbf{this proposal}}, I think there is a PoC construction, but I don't have access permission to it.}
%     \jbel{the full architecture and POC don't exist yet -- there are several streams going in parallel (some people thinking about architecture, some people benchmarking to inform that, some people creating a POC)}
%     \begin{itemize}
%         \item What are the main achievements, main results of our work?
%         \item What are the properties that our scheme satisfies?
%         \item What is the best experiment results and the environment of it?
%     \end{itemize} 
% \end{itemize}
% \subsection{Related work and how they make things right}
% \begin{itemize}
%     \item Before us, are there any solutions for these problems, and what are their pros and cons?
%     \begin{itemize}
%         \item \textbf{Overrall about current approaches is listed in \href{https://docs.google.com/presentation/d/1YROCEHK_t10wo5CukgYWmS1nuYKZi46NJBu-XZ8zXiw/edit?slide=id.p\#slide=id.p}{this presentation} (also in \href{https://docs.google.com/presentation/d/1HqFtSiS2hVHaSS8-u-8iecVFeMehMGBtZJnnbnXj83c/edit?slide=id.g34d4bb36836_0_262\#slide=id.g34d4bb36836_0_262}{this shorter version}), 
%  need to dig deeper to know how they solve these problems above.}
%         \item \textbf{\href{https://docs.google.com/presentation/d/1C4D8zK4gAdafgIEW-2m_qDyyT39gWo0mmFYpwmA8N3M/edit?slide=id.g312b09519cd_0_8\#slide=id.g312b09519cd_0_8}{This slide} also describe similar content but dive deeper in technical, and have the proposal design constraints.}
%         \begin{itemize}
%             \item Google solution: Tradeoffs and Considerations. 
%                 \begin{itemize}
%                     \item Pros: 
%                 \end{itemize}
                
%             \item Microsoft solution: Tradeoffs and Considerations.
%             \item Our zkID:
%                 * preprocessing 
%         \end{itemize}
%     \end{itemize} 
% \end{itemize}


\section{Introduction}
\label{sec:introduction}
%
It is widely recognized that digital identity systems must deliver security, perform reliably on everyday hardware, and preserve user privacy across multiple sessions, and it has also been observed that these requirements become most challenging when deployments move beyond controlled environments to heterogeneous devices, fluctuating networks, and varied implementations~\cite{PoPETS:KruPaiRujKan24}. Within the European Union’s European Digital Identity (EUDI) Architecture \& Reference Framework~\cite{EU:EUDI23}, these aims are stated explicitly through selective disclosure and unlinkability. Achieving security, reliability, and privacy in digital identity systems requires not only robust cryptographic choices but also practical integration into real-world environments. We consider a typical deployment: wallets run on commodity smartphones, verifiers operate within standard web infrastructures, and network conditions vary widely \cite{EU:EUDI23}. In such settings, stable identifiers or protocol patterns can enable unintended correlation of user interactions, compromising privacy. Effective designs must ensure predictable performance under load, minimal resource demands on devices, and low server-side verification costs, while avoiding non-standard cryptographic assumptions that hinder adoption \cite{NIST:Grassi17}. Our zkID system addresses these challenges by prioritizing unlinkability and deployability, as we explore in subsequent sections.

Security and privacy are distinct requirements in digital identity systems: while authenticity and integrity can be ensured, linkability often persists, compromising user privacy~\cite{PoPETS:KruPaiRujKan24,JC:FeiFiaSha88}. We work with the usual roles—issuer, holder, verifier—and with two flows: issuance and presentation. Correlation can arise from deterministic disclosure structures, from identifiers that remain stable (intended or incidental), and from protocol- or transport-level regularities across sessions. Unless policy authorizes correlation, two valid presentations should not be tied back to the same credential; this is the privacy target used here.

Standardized formats such as Selective Disclosure JWT (SD-JWT)~ \cite{IETF:FetYasCam25} and ISO mobile Driving Licence (mDL)~\cite{ISO:18013-5} were designed to minimize disclosed attributes and do so effectively. Even so, disclosure encodings and patterns can be stable enough for cooperating verifiers to correlate presentations over time. Pairing-based credentials (e.g., BBS+\cite{C:BCTV14}) address unlinkability at the credential layer; however, reliance on pairing-friendly curves complicates certification paths and leaves post-quantum migration unsettled in ecosystems that prioritize NIST-standardized primitives~\cite{EC:TesZhu23b}.

A third direction wraps standardized credentials in general-purpose zero-knowledge proofs. This can suppress correlating structure, yet it reintroduces a familiar fork: heavier verification without a common reference string, or a universal setup whose governance is difficult to reconcile with open, multi-party standardization. The question that follows is straightforward to state, although less simple to answer:
\begin{center}
    \textit{Can zero-knowledge be layered over existing, standardized credentials so that unlinkability holds against cooperating verifiers, while practical latency and verifier cost are preserved, and without a universal trusted setup or dependence on non-standard curves?}
\end{center}

For a solution to be acceptable, it should satisfy four requirements. First, deployability: it must interoperate with mDoc/JWT data structures and existing PKI (ECDSA or RSA) without changes to issuer processes or secure elements. Second, performance: proof generation and verification must remain within mobile and web latency budgets, and expensive steps (e.g., issuer-signature checks) should be cached or batched when possible. Third, unlinkability: distinct presentations of the same credential must be non-correlatable under the stated threat model, including colluding verifiers. Finally, transparency and post-quantum agility: no trusted setup and no pairing-friendly curves; rely on standard assumptions (e.g., discrete logarithm) and allow PQ replacements without redesigning the protocol.

\subsection{Our zkID Construction}

Motivated by these requirements, we propose \emph{zkID}, a modular zk-SNARK wrapper aimed at real-world identity workflows. At a high level, the design follows a simple rule: pre-process once, reuse later. Proof work is split into two phases:
\begin{itemize}
    \item \textbf{Prepare.} Run infrequently. The issuer’s ECDSA signature is verified, credential attributes are parsed from SD-JWT (or related formats), and SHA-256 hashes are computed over intended disclosures. The results are committed with Hyrax vector commitments and can be cached per credential~\cite{cryptoeprint:2017/1132}.
    \item \textbf{Show.} Run per presentation: selected attributes or predicates are disclosed, the device-bound nonce signature is validated, and equality is proved against cached commitments from the \emph{Prepare} phase.
\end{itemize}

Our construction leverages Spartan interactive-oracle proofs over the Tom256 elliptic curve, aligning with ECDSA infrastructures (e.g., NIST P-256) and using Hyrax commitments for efficient linking \cite{C:Setty20,cryptoeprint:2017/1132}. Our evaluation considers a standard adversary model with malicious holders, colluding verifiers, and honest-but-curious issuers, deferring quantum adversaries and side-channel attacks to future work. To highlight differences in zkID’s approach, we review related privacy-preserving credential systems, comparing their strategies for unlinkability and deployability with our solution \cite{cryptoeprint:2024/2010,cryptoeprint:2024/2013}.

\subsection{Related Work}

\paragraph{Anonymous Credentials from ECDSA.}
Matteo Frigo and Abhi Shelat (Google) propose an anonymous credential scheme for legacy-deployed ECDSA that preserves deployability on existing infrastructures (e.g., P-256, SHA-256) without issuer or device changes and without trusted setup, addressing reluctance to upgrade infrastructures that only support RSA/ECDSA~\cite{cryptoeprint:2024/2010}. Their design tackles the bottleneck of zero-knowledge proofs over non-NTT-friendly curves by building a proof stack around sum-check and the Ligero argument~\cite{CCS:AHIV17}, introducing specialized ECDSA circuits and efficient Reed–Solomon encodings, and adding a witness-consistency mechanism to keep common witness values aligned across arithmetic over $\mathbb{F}_{p}$ (ECDSA) and $\mathrm{GF}(2^k)$ (SHA-256). As reported, the system generates a proof for a single ECDSA signature in 60 ms and a zero-knowledge proof for the ISO/IEC 18013-5 mDL presentation flow in 1.2 s on mobile devices~\cite[\S5.3,\S6.2]{cryptoeprint:2024/2010}. Device binding follows the mDL live-challenge pattern. The main trade-off is comparatively larger proofs and higher verifier workload than succinct CRS-based SNARKs.

\paragraph{Crescent Credentials.}
Christian Paquin, Guru-Vamsi Policharla, and Greg Zaverucha introduce Crescent to upgrade privacy of existing credentials (JWTs, mDL) with selective disclosure and unlinkability without changing issuance processes~\cite{cryptoeprint:2024/2013}. Crescent achieves fast online presentations with compact proofs, while relying on a two-phase workflow:
\begin{itemize}
  \item Prepare (offline, once per credential): verify issuer signatures, decode and parse the credential into attributes, and produce a Pedersen vector commitment via a Groth16 proof. Reported timings are approximately 27 s for a 2 KB JWT and 140 s for an mDL, with universal-setup parameters of size approximately 661 MB–1.1 GB~\cite[\S4]{cryptoeprint:2024/2013}.
  \item Show (online, per presentation): re-randomize the Groth16 proof and attach a few $\Sigma$-proofs over committed attributes; typical latencies are approximately 22 ms for JWT and 41.2 ms for mDL, with proofs around 1 KB. Optional device binding increases these costs (e.g., about 315 ms Show, about 184 ms verify; proof size about 15 KB)~\cite[\S4]{cryptoeprint:2024/2013}.
\end{itemize}
Crescent exposes a modular committed-attribute interface that admits sub-provers for range checks, credential linking, and binding to session context or device by proving knowledge of a signature under a committed public key. The principal trade-offs are reliance on pairing-based Groth16 with a trusted setup and large universal parameters, and a comparatively heavy (but amortized) offline Prepare phase; in its current form Crescent is not post-quantum secure~\cite{groth2016size}.

\paragraph{zkID.}
zkID is a modular, transparent (no-CRS) zero-knowledge wrapper for standardized credentials, prioritizing unlinkability and deployability. We adopt the same two-phase workflow as Crescent but avoid a universal setup, and we retain ECDSA compatibility as in Google’s design while moving issuer verification to the offline phase to reduce online verifier work. Key distinctions from prior work include:
\begin{itemize}
  \item Transparency and PQ agility: no trusted setup; standard discrete-log assumptions; an architecture designed to swap in PQ vector commitments in future instantiations.
  \item Deployment compatibility: interoperates with mDL/SD-JWT and existing PKI (ECDSA or RSA) without changes to issuer processes or secure elements.
  \item Reduced online burden: relative to prior ECDSA-based designs, issuer-signature verification and parsing move to the offline Prepare phase, reducing online compute.
  \item Avoidance of pairing-based cryptography: unlike pairing-based schemes, zkID avoids pairing-friendly curves.
  \item Verifier cost: reduces verifier work relative to prior ECDSA-based designs under our constraints while preserving unlinkability under the stated threat model.
\end{itemize}

In summary, zkID shows that a modular, standards-aligned zero-knowledge wrapper over existing credential formats can be built. Compared to pairing-based schemes, it avoids dependence on pairing-friendly curves; compared to universal-setup SNARK wrappers, it avoids a trusted setup; and relative to prior ECDSA-based designs, it reduces verifier cost while preserving unlinkability under our model. The construction is transparent and PQC-agile; the current instantiation is classical and not quantum-resistant.

In our evaluation, zkID achieves practical performance on commodity hardware. For a typical credential with ..., the one-time \emph{Prepare} phase completes in ... seconds on a standard laptop (or mobile) (..., ...GB RAM), while the per-presentation \emph{Show} phase takes less than ... ms on a mid-range smartphone. Proof sizes are approximately ... KB for the full protocol, and verifier time remains below ... ms on a web server. These results demonstrate that zkID meets latency and resource targets for mobile and web deployments, while preserving unlinkability and avoiding trusted setup. Detailed benchmark results and further breakdowns will be provided in Section~\ref{sec:experiments}.

Several aspects are left out of scope. We do not provide comprehensive revocation, explicit defenses against hardware side channels, or end-to-end wallet-integration benchmarks, and a concrete instantiation of lattice-based vector commitments is deferred to future work.
According to the Cryptographers' Feedback on the EU Digital Identity’s ARF\footnote{\url{https://github.com/user-attachments/files/15904122/cryptographers-feedback.pdf}}, an Anonymous Credential AC scheme, is a suitable cryptographic primitive to instantiate the new EU Digital Identity Wallet (EUDIW) which is an important step towards developing interoperable digital identities in Europe for the public and private sectors.

Informally speaking, an Anonymous Credential AC scheme allows:
\begin{itemize}
	\item An Identity Provider IP to (possibly blindly\footnote{i.e. the IP does not know the content that it signs, only its provenance is satisfied.}) sign a set of (eligible) attributes for a User U;
	\item The User U can show, only if they hold the signed attributes (a.k.a Unforgebality), usually through a Presentation, to a Relying Party RP such that:
	\begin{itemize}
		\item The RP can verify that the set of attributes (signed by IP) that the User U holds satisfy some condition of their interest (a.k.a Correctness);
		\item The RP cannot learn any \emph{additional}\footnote{We stress that the RP may have obtained some privacy sensitive information prior to this presentation.} information beyond the fact that the condition is satisfied or information that can be inferred from the satisfaction of the condition (a.k.a Zero-Knowledge or Anonymity);
		\item The immediate previous requirement also implies that the RP cannot link the various presentations by the same User U (a.k.a. Unlinkability);
	\end{itemize}
	\item The IP can revoke all or a part of the signed attributes that it has issued to the User U, from upon which, the eligible attributes of the User U are updated, and subsequent presentations have to be based on the new and updated attributes (a.k.a Revocation);
	\item The User U cannot transfer its set of signed attributes to to another User U' (a.k.a Non-transferability).
\end{itemize}

In the aforementioned feedback document, BBS and BBS+\footnote{For BBS, thanks to prior work by the W3C, the Decentralized Identity Foundation, IETF/IRTF, ISO, and other standardization bodies, as well as the availability of open-source software libraries, the EC can develop a standard and reference implementation with only a modest effort. The feedback additionally recommend that the EUDI be designed following the principle of crypto-agility, meaning that its underlying technologies can be upgraded quickly in the future if the need arises.} were promoted as the main candidate, besides that, there have been two independent work from Google and Microsoft that attempted to offer candidate solutions. In this document, we attempt to offer a new candidate, called \textbf{zkID}.

In comparison, these approaches show the current trade-off: systems either reuse existing issuer infrastructure but pay high per-presentation costs, or they achieve fast online proofs at the price of large setups and pairing-based assumptions. Our construction, zkID, aims to combine issuer compatibility with reusable offline work, while remaining transparent and modular.

\subsection{Our zkID}
Let us first outline a reference architecture that represents what an anonymous-credential system would ideally look like if it is to integrate smoothly with current infrastructures. In this model, the issuer is treated as fixed components that continue to use their existing public-key algorithms (such as RSA or ECDSA) and standard credential formats (e.g., JWT or mDL), since it's typically difficult to change once deployed. All additional logic is placed in the user’s wallet and the verifier.
The wallet is expected to operate in two stages: an offline Prepare step, which verifies the issuer’s signature once using standard libraries, parses and normalizes credential attributes (for example, turning a date of birth into an integer age), and commits to those attributes using a binding and hiding commitment scheme (a cryptographic way to lock values so they can later be revealed or proven in restricted form); and an online Show step, which runs per presentation, where the wallet selects only the attributes or predicates required by a relying party’s policy, proves them in zero knowledge against the stored commitments, and includes a fresh device signature over the session challenge to ensure the proof is tied to the holder’s device.
A further requirement is modularity: each major function---issuer signature verification, attribute commitment, predicate proofs, and device binding---should be defined as a separate module with a clear interface. This separation makes it possible to swap the underlying proof engine (for example, using a SNARK today or a post-quantum proof system in the future) without requiring changes to parts of the system that are costly or impractical to modify. The purpose of this modular view is to act as a comparison framework: it outlines how a deployment-friendly anonymous-credential stack could be structured, making it easier to compare proposals by the modules they cover, the constraints they address, and the trade-offs they make.

Our construction works with standardized credentials (e.g., SD-JWT, mDL) and existing PKI (RSA/ECDSA), so issuers do not need to change their issuance pipelines.
The zkID workflow follows the two-phase split in the reference view: a one-time Prepare phase and a per-presentation Show phase.
In Prepare, the wallet verifies the issuer’s signature, parses the credential into normalized messages, computes the associated hashes, and produces two reusable artifacts: (i) zero-knowledge proofs that issuer-side checks and parsing were done correctly, and (ii) Hyrax-style Pedersen vector commitments to a designated message column, supporting efficient proofs over multiple attributes.
In Show, the wallet proves only the verifier’s requested predicates and includes a fresh device-binding signature. To link Prepare and Show without revealing values, the verifier checks equality of commitments across both proofs; the wallet reuses the corresponding randomness for that session.
The proving backend is transparent (no trusted setup). It checks the arithmetic constraints with a sum-check–style protocol and uses a small inner-product check to verify commitment openings. For device binding, we choose a curve whose scalar field matches the device’s signature field (e.g., P-256), so the device signature can be verified directly inside the proof without emulation or field translation. 
In terms of the reference system view, issuer compatibility is preserved, the two-phase reuse is integrated into the workflow, predicates are modular, and there is no trusted setup. The trade-offs are that security currently relies on discrete-log assumptions (not post-quantum) and that commitment equality requires using the same curve across Prepare and Show; the modular interface leaves room to swap in lattice-based commitments when suitable.

\section{Application to EUDI}
\label{sec:appeudi}

% \jbel{Links to read (we can basically copy paste a lot from these):
% \begin{itemize}
% \item EUDI ARF (full) \href{https://eu-digital-identity-wallet.github.io/eudi-doc-architecture-and-reference-framework/latest/}{here}
% \item Discussion of Google/Microsoft pros/cons \href{https://github.com/eu-digital-identity-wallet/eudi-doc-standards-and-technical-specifications/blob/main/docs/technical-specifications/ts4-zkp.md}{here}
% \item Google's IETF draft for libZK \href{https://www.ietf.org/id/draft-google-cfrg-libzk-00.html#name-sumcheck}{here}
% \end{itemize}}

% Why should EUDI consider this report?
% \begin{itemize}
%     \item Does it compatible with current EUDI decision like data format and ecosystems?
%     \item Why governments or organizations should choose this scheme?
%     \item \textbf{Our pros and cons are already shown in other sections, so just mentioned them when we need them in this section}
% \end{itemize}

% Our report is highly relevant to the EUDI's initiatives and demonstrates another viable solution to adding programmable zero-knowledge proofs around digital credential presentation. 

% \paragraph{Issuance.} PID Providers or Attestation Providers would remain oblivious to the use of this scheme, and therefore no changes to the issuance process would be required (which would potentially be very expensive).
% Our solution can handle any of the intended credential data standards (SD-JWT and mDL with standard \href{https://mobiledl-e5018.web.app/ISO_18013-5_E_draft.pdf}{ISO/IEC 18013-5}), 
% as mentioned in \href{https://eu-digital-identity-wallet.github.io/eudi-doc-architecture-and-reference-framework/1.4.0/annexes/annex-2/annex-2-high-level-requirements/}{Annex 2} of the EUDI Architecture Reference Framework.
% Furthermore, due to the generic nature of programmable zkSNARKs, it would be very easy to adapt to any changes in the Issuer's signature scheme in the future (e.g. switching to post-quantum signature schemes) will be easy to take into consideration;
% we would simply modify our zkSNARK circuit to reflect the computation of a new signature verification, without having to come up with a new ad-hoc protocol for a specific signature scheme.  

% \paragraph{Efficiency.} Governments should choose this scheme for its [potential] efficiency due to the nature of how we split up the proofs – by a fixed relation and by a live, presentation-specific relation.
% [TODO: insert concrete benchmarks when we have them].
% Our scheme is also highly modular; one can swap out Spartan for another zkSNARK system that uses polynomial commitment schemes in a modular form, and one can also swap out the polynomial commitment scheme. 
% The benefit of choosing a modular approach is that it is relatively easy to update on future innovations for proof systems that make them more efficient.

% \paragraph{Discussion.} While our scheme is not yet post-quantum secure, its modularity means that it will be relatively easy to swap in modified Ajtai lattice-based commitments as presented by Hwang, Seo, and Song \cite{cryptoeprint:2024/306}.
% Some components of our scheme are not yet standardized. However, the other solutions under consideration from Google and Microsoft also use unstandardized cryptography, and arguably ``more unstandardized'' cryptography. 
% In particular, we do not make any strong assumptions, such as the pairing-based assumptions that Microsoft makes.
% Our zkSNARK uses a Spartan backend, which relies only on the sumcheck and Pedersen commitments.
% Sumcheck is a folklore protocol with information-theoretic security that does not rely on cryptographic assumptions, and Pedersen commitments have been used since 1991 \cite{C:Pedersen91} and only rely on the discrete-log assumption, which standardized ECDSA signatures already rely on. 

% Finally, our team is also actively working on zkSNARK standards, and we believe the long-term solution is not to avoid unstandardized cryptography, but to argue the need for such cryptography in these applications and push forward the corresponding standards. 

Within the EUDI Architecture and Reference Framework~\cite{EU:EUDI23}, the practical question is how to introduce zero-knowledge capabilities without disrupting established roles, formats, and certification paths. This section states how the construction fits that setting and what trade-offs it entails.

\paragraph{Issuance.}
The construction is designed to wrap existing credential encodings rather than replace them. It accommodates SD-JWT and ISO/IEC 18013-5 mDL so that wallets and relying parties retain current disclosure grammars and parsing logic. Issuers (PID Providers and Attestation Providers) remain oblivious to the use of zero-knowledge proofs; no changes to issuance pipelines or device secure elements are required, and issuers keep exclusive control of their private keys. The proof layer is circuit-defined, which allows future issuer-side migrations (for example, a change of signature scheme) to be handled by updating the verification circuit rather than introducing format-specific protocols. The approach interoperates with current PKI based on ECDSA or RSA and does not prescribe a switch of algorithm or hardware.

\paragraph{Efficiency.}
Proving is split into two relations. A fixed relation captures issuer-signature verification, credential parsing, and commitment preparation; it runs infrequently and is amortized per credential. A live, presentation-specific relation captures the disclosures and predicates for a single session; it runs per presentation. This separation aims to keep holder and verifier costs within typical web and mobile budgets and to bound latency on the critical path. The proof system and commitment layer are modular, so improvements in either component can be adopted without redesigning the higher-level flow. In contrast to designs that keep issuer verification online, issuer-signature verification and parsing are moved to the offline step here to reduce presentation-time work at the verifier.

\paragraph{Discussion.}
The present instantiation follows the Spartan line and relies on sumcheck and Pedersen/Hyrax-style commitments rather than pairing-based assumptions; there is no universal trusted setup. The choice avoids pairing-friendly curves and the operational burden of a setup ceremony across many issuers and relying parties. The construction is not post-quantum secure in its current form, but the modular structure leaves a path to replacing the vector-commitment layer with lattice-based alternatives as they mature. Some components have not yet been standardized; this is a shared condition across competing approaches and is called out explicitly in our roadmap.

\paragraph{Summary for EUDI.}
The design aligns with Annex 2 format expectations, requires no changes to issuers, supports current PKI deployments, and separates fixed from presentation-specific work to keep online costs low. It avoids pairing-based assumptions and a universal setup and leaves a path to future cryptographic upgrades without disrupting wallet or issuer operations.


% \jbel{Links to read (we can basically copy paste a lot from these):
% \begin{itemize}
% \item EUDI ARF (full) \href{https://eu-digital-identity-wallet.github.io/eudi-doc-architecture-and-reference-framework/latest/}{here}
% \item Discussion of Google/Microsoft pros/cons \href{https://github.com/eu-digital-identity-wallet/eudi-doc-standards-and-technical-specifications/blob/main/docs/technical-specifications/ts4-zkp.md}{here}
% \item Google's IETF draft for libZK \href{https://www.ietf.org/id/draft-google-cfrg-libzk-00.html#name-sumcheck}{here}
% \end{itemize}}

% Why should EUDI consider this report?
% \begin{itemize}
%     \item Does it compatible with current EUDI decision like data format and ecosystems?
%     \item Why governments or organizations should choose this scheme?
%     \item \textbf{Our pros and cons are already shown in other sections, so just mentioned them when we need them in this section}
% \end{itemize}

\section{Security}
\label{sec:security}

% OUTLINE IN MAIN:

% Give a detailed answer and analysis for:
% \begin{itemize}
%     \item Is the scheme a dishonest majority setting or something else? What happens when the setting is broken?
%     \jbel{\begin{itemize}
%         \item ZK is addressing malicious verifier -- semihonest 
%         \item soundness -- address malicious prover
%         \item trust assumption - verifier is trusted/honest, issuer is trusted/honest
%         \item deniable presentation? -- ask YT
%     \end{itemize}}
        
%     \item If the Issuer needs to update frequently, what if they are disconnected for a while? 
%     \item Place the scheme into a poor network connection, does it still work well and not be vulnerable?
%     \jbel{depends on solution to revocation flow, and also what applications of ID presentation look like (e.g. are the prover and verifier talking through internet channels?)}
% 	\item If it fails during the process, what will happen?
% 	\item If it is not quantum resistant, how do we upgrade it to quantum resistant? -- it is quantum resistant
% \end{itemize}


In our security model, we assume that the Prover is malicious, and that each Verifier is semi-honest, meaning that if the Prover presents a valid proof that they own a credential with some property, the Verifier will grant access to any services for which the property suffices.


\paragraph{Verifier's side} For security on the Verifier's side, our soundness analysis considers the probability that a malicious Prover without real ownership of a valid credential can generate a false proof of ownership. 


\paragraph{Prover's side} For security on the Prover's side, we guarantee that our proofs are zero-knowledge, so that a semi-honest and computationally-bounded Verifier cannot get any additional information about the Prover's credential beyond what is publically revealed in the proof.
In particular, we do not consider the case where the Verifier is malicious during presentation, e.g. where a false Verifier pretends to be an authorized Verifier. The problem of Verifier identity lies outside the scope of this paper.


Furthermore, we assume that Verifiers can collude with each other, i.e. that Verifiers $V_1, \dots, V_N$ that have received proofs $\{\pi_1\}, \dots, \{\pi_N\}$ from a given Prover $P$ can compute functions $f(\pi_1, \dots, \pi_N)$. 
Therefore, we desire the \textbf{unlinkability property}: given $pi_1, \dots, \pi_N$, the Verifiers should not be able to determine whether or not any two of these proofs came from the same Prover $P$.
Note that this requires the Prover to re-randomize each presentation's proof; a static zero-knowledge proof of the same statement, while not revealing private credential information, would still look the same across presentations.
In that case, it may be possible for the Verifier to de-anonymize a Prover by linking their ``anonymous'' activity across presentations and analyzing metadata, e.g. time of presentation.
Fortunately, our scheme is unlinkable due to the re-randomization of proofs between each presentation. By the zero-knowledge property for each presentation, we can simulate the distribution of proofs without knowledge of the witness. 
To simulate an entire set of proofs received by distinct colluding Verifiers, we can independently simulate each proof.  


Finally, as our scheme is currently presented in Section \ref{sec:contribution}, we assume that Verifiers will not collude with Issuers even though they can see the Issuer public key. 
To bypass this assumption and prevent Issuer tracking in the case of malicious Verifiers that collude with Issuers, 
we propose here the maintenance of a trusted Merkle tree on trusted Issuer public keys. 
Then our Prover's circuit would prove knowledge of a valid Issuer signature from some key in the Merkle tree, and the public input/output would just be the Merkle root rather than any specific Issuer public key.

\paragraph{Both Prover and Verifier security} When modelling the Issuer, we assume that the Issuer is trusted during issuance by both the Prover and Verifier, i.e. will not Issue false credentials or sell personal information that is necessarily to obtain about individuals to issue a credential.

\subsection{Other considerations}

Our scheme currently does not require any interaction from the Issuer for credential presentation beyond initial issuance.

However, our scheme does require the use of internet access (without, there will be risks with authorizing someone before their credential can be checked against the current state).
In the case (as presented) where Issuer public key is a public input/output, we assume there is an online registry of trusted Issuer keys that the Verifier can check the proof against.
This requires live internet access in the same way that credit card transactions do, in order to check the most current registry of public keys. 
Even in the case of a Merkle inclusion proof, where the Issuer key is also private, the Verifier would need to check that the public Merkle root matches the trusted root stored online. 
It is possible to store encrypted transactions/credential presentations to be checked later once internet access is restored, in the same way as offline credit card transactions do. 
However, there are necessary risks with this approach; it would be up to the specific vendor and/or service prover what levels of risk can be tolerated from delayed credential authentication.
For example, some service provider (Verifier) may be fine only periodically downloading the current registry of trusted Issuer keys (and/or Merkle roots) and simply checking against their last downloaded version before granting access.


Finally, as mentioned previously, our scheme is not quantum resistant due to the use of Hyrax commitments. 
Again, we believe this is easily fixable with the introduction of modified Ajtai lattice-based commitments, which are post-quantum secure.


\section{Preliminaries - WIP}
\label{sec:preliminaries}
% \ndhy{This section is a work in progress. It will be completed in the next few days.}
% \ndhy{This section is a work in progress. It will be completed in the next few days.}
% \paragraph{Notation.}
% For $n\in\mathbb{N}$, let $[n]=\{1,\dots,n\}$.
% Vectors are in bold, e.g., $\mathbf{a}=(a_1,\ldots,a_\ell)$.
% Concatenation is $\|$.
% For a (possibly randomized) algorithm $\mathsf{Alg}$, we write $y \leftarrow \mathsf{Alg}(x)$ for its output.
% The security parameter is $\lambda$, and $\mathrm{negl}(\lambda)$ denotes a negligible function.
% \subsection{Algebraic setting and basic primitives}
% \label{subsec:algebra}
% Let $\mathbb{F}$ be a prime field of order $q$ and $\mathbb{G}$ a cyclic group of order $q$ with generator $g$, where discrete logarithms are hard.
% We use SHA\mbox{-}256 as $\mathsf{H}:\{0,1\}^\ast\!\rightarrow\!\{0,1\}^{256}$.
% The issuer uses a digital signature scheme $\mathsf{Sig}_I=(\mathsf{KeyGen}_I,\mathsf{Sign}_I,\mathsf{Verify}_I)$ (e.g., ECDSA/RSA), and the device uses $\mathsf{Sig}_D$ for nonce-bound session signatures; both are assumed EUF\mbox{-}CMA.

% \paragraph{Pedersen vector commitments.}
% Fix generators $\mathbf{g}=(g_1,\ldots,g_\ell)$ and $h$ in $\mathbb{G}$.
% For $\mathbf{a}\in\mathbb{F}^\ell$ and $r\in\mathbb{F}$ define
% \[
% \mathsf{Com}(\mathbf{a};r)=\prod_{i=1}^{\ell} g_i^{a_i}\cdot h^{r}\in\mathbb{G}.
% \]
% $\mathsf{Com}$ is perfectly hiding and binding under discrete\mbox{-}log hardness.
% We will need efficient \emph{openings} and \emph{equality proofs} for Pedersen commitments; we realize these via inner-product arguments (IPAs) in the Hyrax framework~\cite{SP:WTSTW18}.

% \paragraph{Hyrax commitments.}
% Hyrax commitments \cite{SP:WTSTW18} allow us to commit to a multilinear extension $\widetilde{P}$ of a function $P:\{0,1\}^{\log n} \rightarrow \mathbb{F}$. 
% Notably, we can express $\widetilde{P}(x_1, \dots, x_{\log n})$ as a vector-matrix-vector product $\vec{v_L} P \vec{v_R}$, where:
% \begin{itemize}
%     \item $P$ is a square matrix of evaluations of $P$ on the $\{0,1\}^{\log n}$ hypercube, represented as $\langle \vec{p_i}\rangle_{i \in \sqrt{n}}$ where $\vec{p_i} \in \mathbb{F}^{\sqrt{n}}$ are columns of $P$
%     \item $\vec{v_L}(x_1, \dots, x_{(\log n)/2}) = \langle \widetilde{\chi}_{(b_1, \dots, b_{(\log n)/2})}(x_1, \dots, x_{(\log n)/2})\rangle _{(b_1, \dots, b_{(\log n)/2})\in \{0,1\}^{(\log n)/2}}$, and 
%     \item $\vec{v_R}(x_{(\log n)/2+1}, \dots, x_{\log n}) = \langle \widetilde{\chi}_{(b_{(\log n)/2+1}, \dots, b_{\log n})}(x_{(\log n)/2}, \dots, x_{\log n})\rangle_{(b_{(\log n)/2}, \dots, b_{\log n}) \in \{0,1\}^{(\log n)/2}}$,
% \end{itemize}
% and $\widetilde{\chi}_{b_1, \dots, b_m} (x_1, \dots, x_m) = \prod_{i \in [m]} \widetilde{\chi}_{b_i}(x_i)$, where
% $\widetilde{\chi_{b_i}}(x_i)$ is the multilinear extension of the function $\chi_{b_i}(x_i): \{0,1\} \rightarrow \mathbb{F}$ given by $\chi_{b_i}(x_i) = x_ib_i + (1-x_i)(1-b_i)$, which equals $1$ if $x_i = b_i$ and $0$ otherwise.

% Then our Hyrax commitment scheme is given by:
% \begin{itemize}
%     \item $\textbf{Setup}(1^{\lambda}, n) = (g_i)_{i \in [\sqrt n]}$ where $g_i \in \mathbb{G}$ are elements of the elliptic curve group $\mathbb{G}$ over which we compute our Pedersen commitments.
%     \item $\textbf{Commit}_h(pp, P) \rightarrow (c,S)$, where 
%     $c = \{c_i\}_{i \in [\sqrt{n}]}$ and $S = \{S_i\}_{i \in [\sqrt{n}]}$, and
%     $(c_i, S_i) \leftarrow Commit_p(pp, \langle \vec{p_i} \rangle)$
%     where $Commit_p$ is the Pedersen vector commitment scheme.
%     \item $\textbf{Eval}_h(pp, c, r, v, n; P, S)$, where $r$ is the opening point and $v$ is the claimed evaluation of $P(r)$, is given by:
%     \begin{itemize}
%         \item Prover and Verifier compute $\vec{v_L} = \vec{v_L}(r_1, \dots, r_{(\log n)/2})$, $\vec{v_R} = \vec{v_R}(r_{(\log n)/2+1}, \dots, r_{\log n})$, 
%     and $C = \sum_{i \in [\sqrt{n}]} \vec{v_L}_i c_i$ (where addition is over $\mathbb{G}$)
%         \item Prover and Verifier engage in an interactive IPA protocol to prove that $P(r)$ is a dot product of $v_R$ and $C$ 
%     \end{itemize} 
% \end{itemize} 

% \subsection{Zero-knowledge proving interface}
% \label{subsec:zk-interface}
% We use a \emph{transparent} general-purpose SNARK in the Spartan family~\cite{C:Setty20}, instantiated over $\mathbb{F}$ for R1CS instances.
% Algorithms are
% \[
% \mathsf{pp} \leftarrow \mathsf{Setup}(1^\lambda),\quad
% \pi \leftarrow \mathsf{Prove}(\mathsf{pp},x,w),\quad
% b \leftarrow \mathsf{Verify}(\mathsf{pp},x,\pi)\in\{0,1\},
% \]
% where $x$ is public input and $w$ the witness.
% (We reference succinct CRS-based systems such as Groth16~\cite{groth2016size} later only for performance comparisons.)

% \paragraph{R1CS instance.}
% An \textit{R1CS instance} is a tuple $(\mathbb{F}, A, B, C, io, n, m)$, 
% where $io$ denotes the public input and output of the instance, 
% $A, B, C \in \mathbb{F}^{n \times n}$, where $m \geq |io| + 1$
% and there are at most $m$ non-zero entries in each matrix. An instance is said to be \textit{satisfiable}
% if there exists a witness $\vec{w} \in F^{n - |io| - 1}$ such that
% $(A \cdot Z) \circ (B \cdot Z) = (C \cdot Z),$
% where $\vec{Z} = (io, 1, \vec{w})$, and $\circ$ is the Hadamard (entry-wise) product.

% \paragraph{} Throughout the paper, assume that we are dealing with sparse R1CS instances, where $m = O(n)$.

% \paragraph{R1CS as a language} We let the language 
% \[
% R_{\text{R1CS}} = \{ \langle x = (\mathbb{F}, A, B, C, io, n, m): \text{$x$ is satisfiable }\rangle \}.
% \]
% The language $R_{\text{R1CS}}$ is NP-complete.

% \paragraph{} A Spartan zero-knowledge proof is an argument of knowledge for the R1CS language.
% The Prover can prove not only that a given instance $x \in R_{\text{R1CS}}$, but knowledge of the corresponding witness $\vec{w}$.   
% At a high level, this application considers R1CS instances that represent the computational structure of ownership of a valid credential, along with any other desired properties about the credential. 
% Thus a valid argument of knowledge implies knowledge of the underlying credential that has the claimed properties, which is necessary for real-world authentication.

% % \subsection{Relations used by \textsf{zkID}}
% % \label{subsec:relations}
% % Let $\mathsf{Parse}$ be a deterministic parser that extracts $(\mathbf{a},\mathbf{s},\mathbf{h},\sigma_I,\mathsf{meta})$ from a token $\tau$ (attributes, salts, digests, issuer signature, metadata), consistent with the chosen format (SD\mbox{-}JWT/mDL).

% % \paragraph{Prepare relation $\mathcal{R}_{\mathrm{prep}}$.}
% % \dots
% % \paragraph{Show relation $\mathcal{R}_{\mathrm{show}}$.}
% % \dots

% \subsection{High-Level Credential Presentation Flow}\label{sec:high-level-flow}

% The following outlines the high-level protocol for a credential presentation.

% \begin{enumerate}
% \item The Prover receives a signed credential from an Issuer to be stored securely in their wallet, issued to the Wallet Secure Cryptographic Device (WSCD) public key.
% \item At presentation time, the Verifier sends over challenge $\texttt{nonce}_V$ for device-binding verification.
% \item The Prover signs the challenge \texttt{nonce} with the public key $p_U$ controlled by their WSCD and specified in their credential.
% \item The Prover computes two separate but linked zero-knowledge proofs $\pi_{\text{\texttt{prepare}}}, \pi_{\text{\texttt{show}}}$ which together cover the following statements: SD-JWT parsing, verification of the SD-JWT Issuer signature, 
% proper disclosures and/or arbitrary predicates on the disclosures, and device-binding. 
% (i.e.\ checks the \texttt{nonce} signature against their public key); then sends $\pi_{\text{\texttt{prepare}}}, \pi_{\text{\texttt{show}}}$ to the Verifier.
% \item The Verifier verifies $\pi_{\text{\texttt{prepare}}}, \pi_{\text{\texttt{show}}}$ independently, and also verify that they are linked; grants Prover access to some service based on their credential disclosures.
% \end{enumerate}

% \subsection{Security model}
% \label{subsec:security-model}

% In our security model, we assume that the Prover is malicious, and that each Verifier is semi-honest, meaning that if the Prover presents a valid proof that they own a credential with some property, the Verifier will grant access to any services for which the property suffices.


% For security on the Verifier's side, our soundness analysis considers the probability that a Prover without real ownership of a valid credential can generate a false proof of ownership. 


% For security on the Prover's side, we guarantee that our proofs are zero-knowledge, so that a semi-honest computationally bounded Verifier cannot get any additional private information about the Prover's credential given the proof, beyond what is publically revealed in the proof.
% In particular, we do not consider the case where the Verifier is malicious, where e.g. a false Verifier pretends to be an authorized Verifier. The problem of Verifier identity lies outside the scope of this paper.


% Futhermore, we assume that Verifiers can collude, i.e. that Verifiers $V_1, \dots, V_N$ that have receieved proofs $\{\pi_1\}, \dots, \{\pi_N\}$ from a given Prover $P$ can compute functions $f(\pi_1, \dots, \pi_N)$. 
% Therefore, we desire the \textbf{unlinkability property}: given $pi_1, \dots, \pi_N$, the Verifiers should not be able to determine whether or not any two of these proofs came from the same Prover $P$.
% Note that this requires the Prover to re-randomize each presentation's proof; a static zero-knowledge proof of the same statement, while not revealing private credential information, will still look the same.
% It is possible that Verifier's can effectively de-anonymize a Prover by linking their anonymous activity across presentations and analyzing corresponding metadata, e.g. time of presentation.

\paragraph{Notation}
For $n \in \mathbb{N}$ we write $[n]=\{1,\ldots,n\}$. Bold letters denote vectors, e.g., $\mathbf{m}=(m_1,\ldots,m_n)$. Concatenation is written $\|$. The security parameter is $\lambda$; $\mathsf{negl}(\lambda)$ denotes a negligible function. For a (possibly randomized) algorithm $\mathsf{Alg}$, we write $y \leftarrow \mathsf{Alg}(x)$ for its output on input $x$.

There are three roles:
\begin{itemize}
  \item The \emph{issuer} $I$ signs credentials with a long-term key pair $(SK_I,PK_I)$ (e.g., ECDSA P--256 or RSA).
  \item The \emph{prover} $P$ is the holder’s wallet, which stores credentials and generates proofs. 
  \item The \emph{verifier} $V$ is the relying party that checks proofs against a policy.
\end{itemize}
For device binding, the prover’s secure element holds an additional signing key pair $(SK_D,PK_D)$ used only to sign fresh per-session challenges.

\paragraph{Credentials}
A credential is a standardized signed object $S$ (e.g., SD--JWT~\cite{IETF:FetYasCam25} or mDL~\cite{ISO:18013-5}). 
Parsing maps $S$ into an ordered vector of attributes
\[
  \mathbf{m}=(m_1,\ldots,m_n).
\]
Non-numeric fields (strings, dates) are encoded injectively into integers. 
The resulting integers are interpreted in a prime field $\mathbb{F}=\mathbb{F}_q$ chosen for the proof backend. 
For each attribute $m_i$ we sample a salt $s_i \leftarrow \mathbb{F}$ and compute
\[
  h_i = \mathsf{H}(m_i \,\|\, s_i),
\]
where $\mathsf{H}$ is instantiated as SHA--256. 
The issuer’s signature is
\[
  \sigma_I = \mathsf{Sign}_{SK_I}(h_1,\ldots,h_n),
\]
verified under $PK_I$.

\paragraph{Commitments and Proof Interface}
To support selective disclosure without revealing raw attributes, the wallet commits to $\mathbf{m}$ using Pedersen vector commitments. 
Let $\mathbb{G}$ be a cyclic group of prime order $q$ with public generators $(g_1,\ldots,g_n,h)$ derived from a domain-separated seed. 
For randomness $r \leftarrow \mathbb{F}$, the commitment is
\[
  C = \prod_{i=1}^n g_i^{\,m_i}\cdot h^{\,r}\;\in\;\mathbb{G}.
\]
Under discrete-logarithm hardness in $\mathbb{G}$, these commitments are computationally binding; they are also perfectly hiding. 
To avoid linkability, the wallet re-randomizes $r$ across sessions. 
If it precomputes several reusable commitments, we index them $(C^{(j)},r^{(j)})$; both offline and online proofs in a session reference the same $C^{(j)}$, allowing the verifier to link the phases without learning $\mathbf{m}$.

Credential use is captured by two relations:
\begin{itemize}
  \item \emph{Prepare (offline).} Once per credential, the wallet verifies $\sigma_I$ under $PK_I$, parses $S$ into $\mathbf{m}$, computes digests $\{h_i\}$, derives a commitment $C^{(j)}$, and produces a reusable proof
  \[
    \pi_{\mathrm{prep}}^{(j)} : \;\; \text{``$S$ parses to $\mathbf{m}$, $\sigma_I$ verifies, and $C^{(j)}$ commits to $\mathbf{m}$''.}
  \]
  \item \emph{Show (online).} For each presentation, the verifier sends a challenge $\mathit{ch}$. The device signs it as $\sigma_{\mathit{ch}}=\mathsf{Sign}_{SK_D}(\mathit{ch})$. The wallet proves that all predicates in the verifier’s policy hold with respect to $C^{(j)}$ and incorporates $\sigma_{\mathit{ch}}$:
  \[
    \pi_{\mathrm{show}}^{(j)} : \;\; \text{``policy holds for $C^{(j)}$, and the session is bound via $\sigma_{\mathit{ch}}$''.}
  \]
  The verifier checks $\pi_{\mathrm{prep}}^{(j)}$, $\pi_{\mathrm{show}}^{(j)}$, their consistency on $C^{(j)}$, and verifies $\sigma_{\mathit{ch}}$ under $PK_D$.
\end{itemize}

This split amortizes heavy work (signature verification, parsing, commitment) offline, leaving online interaction to short proofs plus one device signature.

\paragraph{Predicates and Policies}
A \emph{predicate} is a Boolean function $f(\mathbf{m}[S]) \in \{0,1\}$ over a subvector indexed by $S \subseteq [n]$. 
Typical predicates include range checks ($m_i \ge 18$), equality or membership tests (e.g., $m_i$ equals a country code), and cross-credential comparisons. 
A \emph{policy} is a finite set of predicates chosen by the verifier. 
In each session, the wallet proves in zero knowledge that all predicates in the policy hold with respect to $C^{(j)}$, revealing only what the policy requires. 
Because predicates are modular, the proving backend can be swapped (e.g., from a SNARK to a post-quantum argument system) without changes to issuer infrastructure.

% \paragraph{Cryptographic Backend and Assumptions}
% Proofs are instantiated with Spartan~\cite{C:Setty20}, a transparent argument system based on sum-check protocols with zero-knowledge modifications~\cite{ZXZS19}. 
% We combine this with Hyrax-style inner-product arguments (IPAs)~\cite{SP:WTSTW18} for openings and equality checks on Pedersen commitments. 
% This avoids a trusted setup and supports reusable offline computation while keeping the interface modular: the backend can be swapped as post-quantum candidates mature.

We assume the issuer is honest and operates standard PKI. Verifiers are semi-honest: they check proofs correctly but may collude to compare transcripts. 
Unlinkability relies on re-randomization of commitments, so the only stable value within a session is $C^{(j)}$, intentionally shared between \emph{Prepare} and \emph{Show}.


% Give a detailed answer and analysis for:
% \begin{itemize}
%     \item Is the scheme a dishonest majority setting or something else? What happens when the setting is broken?
%     \jbel{\begin{itemize}
%         \item ZK is addressing malicious verifier -- semihonest 
%         \item soundness -- address malicious prover
%         \item trust assumption - verifier is trusted/honest, issuer is trusted/honest
%         \item deniable presentation? -- ask YT
%     \end{itemize}}
        
%     \item If the Issuer needs to update frequently, what if they are disconnected for a while? 
%     \item Place the scheme into a poor network connection, does it still work well and not be vulnerable?
%     \jbel{depends on solution to revocation flow, and also what applications of ID presentation look like (e.g. are the prover and verifier talking through internet channels?)}
% 	\item If it fails during the process, what will happen?
% 	\item If it is not quantum resistant, how do we upgrade it to quantum resistant? -- it is quantum resistant
% \end{itemize}

\section{Our zkID}
\label{sec:contribution}
% \textit{main deliveries: 1. describe zkID; 2. the detailed construction}
At a high-level, we propose a generic zkSNARK wrapper over a credential, which will either in SD-JWT format or the mDL data format specified in standard \href{https://mobiledl-e5018.web.app/ISO_18013-5_E_draft.pdf}{ISO/IEC 18013-5}). The backend proving system we use will be a combination of Spartan and Hyrax commitments, with modifications to be zero-knowledge. 

There are two (2) key ideas to highlight within our proposed architecture:

\begin{itemize}
    \item \textbf{Pre-processing batches of re-randomized proofs} of issuer-signature and credential-parsing, to re-use across each new presentation. We call this the ``prepare'' relation.
    \item \textbf{Committing to the credential disclosures with Hyrax commitments} \cite{cryptoeprint:2017/1132}, which allows us to re-use (or ``link'') witnesses across circuits ``for free''
\end{itemize}

For the first item, we note that pre-processing proofs for the ``prepare'' relation is possible because the relation is independent of the presentation, including the choice of disclosures or predicate proofs. Further optimizations can likely be made to only parse the discosures of certain attributes within the credential if it is known that the Wallet User will rarely or never present certain disclosures to external verifiers.

The second items differs from the linking circuits approach that Google uses \cite{cryptoeprint:2024/2010}. Note that Google verifiably computes a hiding and binding MAC of the shared witnesses as a public output of the circuit, which the verifier checks consistency of in plain. Although this is only a few linear relations, it requires the prover to also commit to their portion of the key to prevent forging. We instead simply manually separate out the disclosures $m_1, \dots, m_n$ into a designated column when committing to the witness, which is already needed to prove the circuit relation, and the verifier checks consistency of these commitments when verifying each circuit's proof.

Throughout the remainder of this section, we refer to the EUDI's Wallet User as the ``Prover'', the Relying Party as the ``Verifier'', and the EUDI Attestation Authority (EAA) as the ``Issuer''. Below, we briefly detail a high-level flow of the interaction between the Issuer, Prover, and Verifier.

\subsection{High-Level Credential Issuance and Presentation Flow}\label{sec:high-level-flow}

\begin{enumerate}
\item The Prover receives a signed credential from an Issuer to be stored securely in their wallet, issued to the Wallet Secure Cryptographic Device (WSCD) public key.
\item At presentation time, the Verifier sends over challenge $\texttt{nonce}_V$ for device-binding
\item The Prover signs the challenge \texttt{nonce} with the public key $p_U$ controlled by their WSCD and specified in their credential
\item The Prover computes two separate but linked zero-knowledge proofs $\pi_{\text{``prepare''}}, \pi_{\text{``show''}}$ which together cover the following statements: SD-JWT parsing, verification of the SD-JWT Issuer signature, 
proper disclosures and/or arbitrary predicates on the disclosures, and device-binding 
(i.e.\ checks the \texttt{nonce} signature against their public key); then sends $\pi_{\text{``prepare''}}, \pi_{\text{``show''}}$ to the Verifier.
\item The Verifier verifies $\pi_{\text{``prepare''}}, \pi_{\text{``show''}}$ independently, and also verify that they are linked; grants Prover access to some service based on their credential disclosures.
\end{enumerate}

\subsection{Underlying ZK Circuit}

In this section, we describe our high-level ZK circuit $C$ underlying the knowledge the prover actually needs to present to the verifier.
Throughout, we refer to the Issuer with variable $I$, Prover with variable $P$, and Verifier with variable $V$ (e.g. in subscripts). 

We will detail the ZK wrapper around the SD-JWT credential as an example, but the protocol is analogous for other credential formats. 
Throughout, we will refer to digests as ``message hashes'' or just ``hashes'', and disclosures as ``messages''.

We define a circuit $C$ for proving ownership of an anonymous credential. 
We let our witness $w = S$ be the SD-JWT credential consisting of messages $\{m_i\}_{i=1}^N$, hash salts $\{s_i\}_{i=1}^N$, 
hashes $\{h_i\}_{i=1}^N$, and an Issuer signature $\sigma_I = \sigma(h_1, \dots, h_N; SK_I)$. 
Without loss of generality, we assume that the Prover's public key $PK_P$ is contained in message $m_1$ of the credential and indexable as $m_1[1]$.
We let our instance $x = (PK_I, \{f_i\}_{i=1}^K, \{p_i\}_{i=1}^K)$ contain the Issuer's public key $PK_I$, functions $f_i$ over the messages, 
output predicates $p_i$, and finally the nonce signature $\sigma_{\text{nonce}}$ for proving device-binding. 
The $f_i$ can be arbitrary statements we wish to prove about the messages. For example, one could define a function $f_i(m_1, \dots, m_N) = m_1$ would output a predicate that is just the disclosure of message $m_1$.

\begin{mdframed}[style=zkprotocolwithheader, frametitle=Underlying ZK Circuit $C$ for Verifiable Credential]

We define circuit $C(x = (PK_I, \{f_i\}_{i=1}^K, \{p_i\}_{i=1}^K), w_C = (S))$ as follows:

\begin{enumerate}
\item Assert $\text{parse}_{\text{SD-JWT}}(S) = (\{m_i\}, \{s_i\}, \{h_i\}_i, \sigma_I)$ parsing of the SD-JWT into messages $\{m_i\}_{i=1}^N$, message salts $\{s_i\}_{i=1}^N$, hashes $\{h_i\}_{i=1}^N$ and Issuer signature $\sigma_I$.
\item Assert $h_i = \text{SHA256}(m_i, s_i) \quad \forall i \in [n]$, i.e.\ that messages hashes correspond to messages and salts
\item Assert $p_i = f_i(m_1, \dots, m_n) \quad \forall i \in [n]$ , i.e.\ correct evaluation of the predicates
\item Assert $\text{ECDSA.verify}(\sigma_I, PK_I) = 1$, i.e.\ the credential signature verifies under the Issuer public key
\item Assert $\text{ECDSA.verify}(\sigma_{\text{nonce}}, m_1[1]) = 1$, i.e.\ that the live nonce signature corresponds to the public key the credential was issued to
\end{enumerate}

\end{mdframed}

\subsection{Pre-processing and linking proofs}

The main speedups from our proving system will come from splitting our high-level circuit $C$ above into two (2) circuits for different relations regarding the digital credential, 
namely a \texttt{prepare} and a \texttt{show} relation, analogous to Microsoft's Crescent Credentials \cite{cryptoeprint:2024/2013}. 
This is advantageous because proofs of the \texttt{prepare} relation can be computed a-priori for any credential, as they do not depend on the claim being proved at presentation time. 
Pre-computing these proofs will save significant time per presentation, and reduce the performance bottleneck to that of proving the \texttt{show} relation.

One issue that arises is the need to ensure consistency of witnesses across these separate circuits, or what we call ``linking proofs''. 
At a high level, as opposed to Google's MAC approach \cite{cryptoeprint:2024/2010}, the prover sends Hyrax commitments to the parts of the witness re-used across circuits, 
which ends up being just the raw messages $\{m_i\}_i$. 
The verifier can then check consistency of these witnesses across the circuits $C_i$ by comparing the Hyrax commitments they receive as part of the proof. 
This approach gives us linking ``for free'', as the Prover alreadys needs to compute these Hyrax commitments as part of the proof.

We also want to highlight that we are no longer splitting up circuits by their field operations 
(e.g. SHA256 attestations over a binary extension field and an ECDSA verifications over a prime field), but rather we only split up our circuit by pre-processing and per-presentation relations. 
In particular, the circuit for \texttt{prepare} will necessarily involve wrong-field arithmetic by including both the SHA256 hashes and the Issuer ECDSA signature verification. 
However, because of our ability to pre-compute proofs of the ``prepare'' relation, the more important thing is to choose curves that allow for i) efficient show relations and ii) linking the prepare and show relation. 
Since the verifier can only check equality of Hyrax Pedersen commitments defined over the same curve, we must use the same curve for proving both the \texttt{prepare} and \texttt{show} relations.
Thus we choose a curve with a scalar field equivalent to the base field of the nonce signature $\sigma_{\text{nonce}}$ for efficient signature verification. 
Because most Hardware Security Modules (HSMs) sign over the P256 curve, we choose the Tom256 (T256) curve for our backend, which has scalar field equivalent to the base field of P256. 

We now detail each of the two (2) relations/circuits below.

\subsubsection{The \texttt{prepare} relation:}

The \texttt{prepare} relation checks the validity of issuer signature, parses the SD-JWT, and verifies all the message hashes, none of which depend on the specific presentation. 
Thus, the prover will periodically pre-compute and store a batch of re-randomized proofs of the prepare relation. 
These proofs will utilize Hyrax Pedersen vector commitments as introduced above in order to link the proofs of \texttt{prepare} relation to the \texttt{show}. 

\begin{mdframed}[style=zkprotocolwithheader, frametitle=Circuit $C_1$ for the \texttt{prepare} relation]

We define circuit $C(x = (PK_I), w_i = S, w = (\{m_i\}_{i=1}^N))$ as follows:

\begin{enumerate}
\item Assert $\text{parse}_{\text{SD-JWT}}(S) = (\{m_i\}, \{s_i\}, \{h_i\}_i, \sigma_I)$ parsing of the SD-JWT into messages $\{m_i\}_{i=1}^N$, message salts $\{s_i\}_{i=1}^N$, hashes $\{h_i\}_{i=1}^N$ and Issuer signature $\sigma_I$.
\item Assert $h_i = \text{SHA256}(m_i, s_i) \quad \forall i \in [n]$, i.e.\ that messages hashes correspond to messages and salts
\item Assert $\text{ECDSA.verify}(\sigma_I, PK_I) = 1$, i.e.\ the credential signature verifies under the Issuer public key
\end{enumerate}

\end{mdframed}

To produce zkSNARK proofs for this circuit $C_1$, the prover will proceed in two phases:
\begin{enumerate}
\item \texttt{prepareCommit}: Computes a Hyrax commitment $com_1 = com(m_1, \dots, m_N; r_1)$ to the witness column containing message hashes $\{m_i\}_{i \in [N]}$ using initial randomness $r_1$. 
\item \texttt{prepareBatch}: 
    \begin{enumerate}
        \item Re-randomizes this initial commitment to get a batch of commitments $com_i = com(m_1, \dots, m_N; r_i) = com_1 \cdot g_{N+1}^{r_i-r_1}$ for all $i \in [m]$, 
        where our batch size $m$ depends on the frequency of proof generation and demand for the credential
        \item Continues the Spartan sumcheck IOP on each $com_i$ to produce a batch of proofs $\{\pi_i\}$ for $i \in [m]$ of the ``prepare'' relation.
    \end{enumerate}
\end{enumerate}
    
The prover will run \texttt{prepareBatch} periodically to both generate re-randomized commitments $com_i$ and store their randomness $r_i$ (for linking purposes), as well as generate and store batches of issuer-signature proofs $\pi_i$ that can be used for each presentation.

\subsubsection{The \texttt{show} relation:}

At a high-level, our show relation will i) verifiably compute any functions $f_i$ over the SD-JWT messages (such as disclosures, range checks, etc.),
and ii) check that the credential belongs to the prover's device (also known as proof of ``device-binding''). 
As part of device-binding, the prover will sign a verifier \texttt{nonce} outside of the circuit, as outlined in flow \ref{sec:high-level-flow}. 
Let us denote this signature by $\sigma_P = \sigma(\text{nonce}; SK_P)$

Again, we will use T256 curve for our backend proving system so that the holder in-circuit signature verification can proceed naturally in the right field.

\begin{mdframed}[style=zkprotocolwithheader, frametitle=Circuit $C_2$ for the \texttt{show} relation]

We define circuit $C_2(x = (\{f_i\}_{i=1}^K, \{p_i\}_{i=1}^K), w = \{m_i\}_{i=1}^N)$ as follows:

\begin{enumerate}
\item Assert $p_i = f_i(m_1, \dots, m_n) \quad \forall i \in [n]$, i.e. correct evaluation of the predicates
\item Assert $\text{ECDSA.verify}(\sigma_{\text{nonce}}, m_1[1]) = 1$, i.e. that the live nonce signature corresponds to the public key the credential was issued to
\end{enumerate}

\end{mdframed}

As part of the proof for presentation $i \in [m]$, the prover computes the Hyrax commitment over the Tom256 curve $com(m_1, \dots, m_N; r_i)$, 
notably with the same randomness $r_i$ used during the \texttt{prepareBatch} process, in the proof generation. 
The verifier will check that this equals the re-randomized commitment $com_i$ from proof $\pi_i$ for circuit $C_1$. 

\subsection{Adding ZK to Spartan}

Our construction uses Circom in the frontend to compile our computation into an R1CS instance, witness pair $(x=(\mathbb{F}, A, B, C, io, n, m), \vec{w})$, 
which we then feed into the Spartan IOP coupled with Hyrax-style Pedersen polynomial commitments.

Recall that our R1CS relation looks like the following:
$$
(A \cdot \vec{Z}) \circ (B \cdot \vec{Z}) - (C \cdot \vec{Z}) = 0
$$
where our square matrices $A, B, C$ have size $n$.

Recall that Spartan converts an R1CS relation into the following zero-check:
$$
\sum_{x \in \{0,1\}^{\log n}} \widetilde{eq}(x, \tau) 
\bigg[\bigg(\sum_{y \in \{0,1\}^{\log n}} \widetilde{A}(x,y) \widetilde{Z}(y)\bigg)
\bigg(\sum_{y \in \{0,1\}^{\log n}} \widetilde{B}(x,y) \widetilde{Z}(y)\bigg)
- \bigg(\sum_{y \in \{0,1\}^{\log n}} \widetilde{C}(x,y) \widetilde{Z}(y)\bigg)\bigg] = 0
$$
for some random challenge $\tau \in \mathbb{F}$

There are two components in Spartan that we need to modify to be ZK. 
The first is making the sumchecks ZK. 
The second is to ensure that the opening $\widetilde{Z}$ using the commitment to $\vec{Z}$ does not leak information about our witness $\vec{w}$.

\subsubsection{Adding ZK to sumcheck}

The Spartan protocol consists of several sumchecks in parallel. There are various existing techniques to make sumcheck ZK. 
We employ one using similar methods as in Zhang et. al. \cite{cryptoeprint:2019/1482}, which adds random pads to the sumcheck transcript.

In particular, suppose at each round $i$ of the sumcheck protocol, the prover sends over $s_i(X) := \sum_{} F(r_1, \dots, r_{i-1}, X, x_{i+1}, \dots, x_m)$ 
where $r_i$ is the Verifier challenge sent for round $i$. Then instead of having the verifier check the sumcheck, the prover will prove in ZK that the unpadded transcript satisfies the verifier's (linear) checks.
To do this, the prover will need to commit to the random pads ahead of time. 
Then, as long as the Fiat-shamir challenges is generated from the transcript including these random pad commitments, the prover cannot simply lie about the pads to satisfy the sumcheck relation.

\begin{mdframed}[style=zkprotocolwithheader, frametitle=Adding ZK to sumcheck]
    \begin{enumerate}
        \item Prover commits to pads $r_i(X)$ for all $i \in [\log n]$. These are linear polynomials, and can thus be represented by its $2$ coefficients $r_i[0]$ and $r_i[1]$.
        \item Instead of sending partial sums 
        \begin{equation*}
        s_i(X) := \sum_{(x_{i+1}, \dots, x_n) \in \{0,1\}^{n-i}} F(r_1, \dots, r_{i-1}, X, x_{i+1}, \dots, x_n)
        \end{equation*}
        for each round of sumcheck, the Prover sends polys $s'_i(X) = s_i(X) + r_i(X)$, essentially a one-time-padded transcript.
        \item It suffices to show the following linear relation in zero-knowledge, 
        \begin{equation*}
        \left[\begin{array}{c|c|c}
        A & B & C
        \end{array}\right]
        \left[\begin{array}{c}
            \vec{S} \\
            \hline
            \vec{R} \\
            \hline
            C \\
            F(r) \\
        \end{array}\right]
        = 
        \vec{0}
        \end{equation*}
        where $\vec{S} = [s'_1[0], s'_1[1], \dots, s'_n[0], s'_n[1]]^T$ 
        is the column vector of sumcheck transcripts such that $s'_i = s'_i(X) = s'_i[0]X + s'_i[1]$, and
        where $\vec{R} = [r_1[0], r_1[1], \dots, r_n[0], r_n[1]]^T$ 
        is the column vector of random pads, and
        where matrices $A,B,C$ are given by
        \begin{equation*}
            A = 
            \begin{bmatrix}
                1 & 2 & 0 & 0 & 0 & 0 & 0 & \cdots & 0 & 0 & 0 & 0 \\
                -r_1 & -1 & 1 & 2 & 0 & 0 & 0 & \cdots & 0 & 0 & 0 & 0 \\
                0 & 0 & -r_2 & -1 & 1 & 2 & 0 & \cdots & 0 & 0 & 0 & 0 \\
                \vdots & \vdots & \vdots & \vdots & \vdots & \vdots & \vdots & \ddots & \vdots & \vdots & \vdots & \vdots \\
                0 & 0 & 0 & 0 & 0 & 0 & 0 & \cdots & 0 & -r_{n-1} & -1 & 1 & 2 \\
                0 & 0 & 0 & 0 & 0 & 0 & 0 & \cdots & 0 & 0 & 0 & r_n & 1 \\
            \end{bmatrix}
        \end{equation*}
        \begin{equation*}
            B = 
            \begin{bmatrix}
                -1 & -2 & 0 & 0 & 0 & 0 & 0 & \cdots & 0 & 0 & 0 & 0 & -1 & 0 \\
                r_1 & 1 & -1 & -2 & 0 & 0 & 0 & \cdots & 0 & 0 & 0 & 0 & 0 & 0 \\
                0 & 0 & r_2 & 1 & -1 & -2 & 0 & \cdots & 0 & 0 & 0 & 0 & 0 & 0\\
                \vdots & \vdots & \vdots & \vdots & \vdots & \vdots & \vdots & \ddots & \vdots & \vdots & \vdots & \vdots & \vdots & \vdots \\
                0 & 0 & 0 & 0 & 0 & 0 & 0 & \cdots & 0 & r_{n-1} & 1 & -1 & -2 & 0 & 0\\
                0 & 0 & 0 & 0 & 0 & 0 & 0 & \cdots & 0 & 0 & 0 & -r_n & -1 & 0 & -1\\
            \end{bmatrix}
        \end{equation*}
        \begin{equation*}
            C = 
            \begin{bmatrix}
                -1 & 0 \\
                0 & 0 \\
                0 & 0\\
                \vdots & \vdots \\
                0 & 0\\
                0 & -1\\
            \end{bmatrix}
        \end{equation*}

    \item The Prover computes a public random challenge $\alpha$ (e.g. hashing the transcript) and compresses the relation into a dot product
    \begin{equation} \label{eqn:rlc-vector-reln}
        [\vec{u}]
        \left[\begin{array}{c}
            \vec{S} \\
            \hline
            \vec{R} \\
            \hline
            C \\
            F(r) \\
        \end{array}\right]
        = 
        \vec{0}
    \end{equation}
    where $\vec{u} = [1, \alpha, \alpha^2, \dots, \alpha^n] \left[\begin{array}{c|c|c}
        A & B & C
    \end{array}\right]$ is a random linear combination of the rows.
    \item The Prover and Verifier engage in a proof-of-dot product protocol to prove the relation above, such as an Inner Product Argument used in Bulletproofs \cite{cryptoeprint:2017/1066}
    \end{enumerate}
\end{mdframed}

\subsubsection{Adding ZK to the opening of $\widetilde{Z}$}\label{subsubsec:zk-open-Z}

In order to add ZK to the opening of, we simply append $Z$ with random pads, which is equivalent to assigning random evaluations on the remaining points in the hypercube $\{0,1\}^{\log n}$.
so that $\widetilde{Z'}(x_1, \dots, x_{\log n}) = \widetilde{Z}(x_1, \dots, x_{\log n}) + \sum_{x \in P}eq(r, x)Z(x)$,
where $P$ is the set of points in $\{0,1\}^{\log n}$ that used to be assigned $0$ by default, but were filled with random pads. Notice that if any of $Z(x)$ are random, then $\sum_{x \in P}eq(r, x)Z(x)$ is random.
If $n=|Z|$ is not already a power of $2$, then we can simply fill at least one of the remaining evaluations on $\{0,1\}^{\log n}$ with a single random pad. 
If $n=2^m$, we can add another dimension to the hypercube of evaluations of $Z$. 

    % 1. Prover commits to pads r_i(X) (just linear polys)
    % 2. 
    % 3. Prover and verifier engage in mini SNARK to prove linear relations (using a ZK dot product)
    % 4. Prover has to show opening - using a zk dot product (e.g. IPA)

% for i = 1: 
% linear constraint s'_i(0)-r_i(0) + s'_i(1)-r_i(1) = C
% for i = 2 to m (where m = log n)
% linear constraint s'_i(0)-r_i(0) + s'_i(1)-r_i(1) = s'_{i-1}(r_{i-1})-r'_{i-1}(r_{i-1})

% check s'_m(r_m)-r_i(r_m) = f_r where f_r is the claimed value of f(r_1, ... r_m)
% in a separate zkproof, check the opening of f(r_1, ... r_m) from the commitment of f with an IPA

% can use the IPA extractor to construct the snark witness extractor 
% IPA proof of knowledge soundness is in the bulletproof paper Appendix A: (https://eprint.iacr.org/2017/1066.pdf) 
% IPA extractor - proves you know/can get the vectors satisfying to the product relation, 
% which means can extract pads such that, when subtracted from sumcheck transcript, creates valid sumcheck transcript
% which is equivalent to extracting a valid transcript for spartan sumchecks, 
% which is equivalent to knowing witness passing the relation.

\subsection{Cost analysis}

The following section computes the Prover and Verifier costs of Spartan instantiated with Hyrax Pedersen commitments on R1CS instances $(x = \big(\mathbb{F}, A, B, C, io, n, m),\, \vec{w})$, 
where $io$ denotes the vector of public inputs/outputs, $n = |\vec{w}|$ is the dimension of our matrices, 
and $m$ is the number of nonzero entries in our matrices $A,B,C$. We let $Z = (\vec{w}, io, 1)$. 
It is often reasonable to assume that our R1CS matrices are sparse, i.e. $m = O(n)$. However, we present the costs below independent of this assumption.

\begin{itemize}
    \item \textbf{Prover time}: (1) $O(m)$ to generate sumcheck transcript, (2) $O(m)$ to evaluate MLEs of $A,B,C$, (3) $O(n)$ to commit to the MLE of $Z$ (computing $\sqrt{n}$ MSMs of size $\sqrt{n}$) and opening the MLE of $Z$, for a total cost of $O(m)$.
    \item \textbf{Proof length}: (1) $O(\log n) \cdot |\mathbb{F}|$ length of the sumcheck transcript, (2) $O(\sqrt{n}) \cdot |\mathbb{G}|$ length commitment to MLE of $Z$, (3) $O(\log n) \cdot |\mathbb{G}|$ length of argument opening MLE of $Z$, for a total length of $O(\sqrt{n})$ group or field elements.
    \item \textbf{Verifier time}: (1) $O(\log n)$ to verify sumcheck transcript, (2) $O(m)$ to evaluate the MLEs of $A,B,C$ (with sparse commitment scheme and memory checking), (3) $O(\sqrt{n})$ to open the MLE of $Z$, for a total of $O(m + \sqrt{n})$.
\end{itemize}

With the ZK modifications to Spartan, we can see the asymptotic costs remain the same, as follows:

\begin{itemize}
    \item \textbf{Prover time}: additionally computes $O(\log n)$ constant-size commitments to the sumcheck transcript pads $r_i$, and $O(\log n)$ engages in new sumcheck relation IPA (or some other ZK dot product argument) for vector of length $O(\log n)$, which still gives $O(m)$ prover work.
    \item \textbf{Proof length}: sumcheck and openings are the same length but just padded, but added on $O(\log n)$ size $|\mathbb{G}|$ commitments, and a length $O(\log\log n)$ sumcheck relation IPA proof, which still gives a proof length of $O(\sqrt n)$ group or field elements.
    \item \textbf{Verifier time}: no longer needs to do $O(\log n)$ (sumcheck), but still needs $O(m)$ (evaluating MLEs of $A,B,C$) + $O(\sqrt{n})$ (opening MLE of Z) + $O(\log n)$ for sumcheck relation IPA verification, which still gives $\Rightarrow O(m + \sqrt{n})$ runtime.
\end{itemize}

\subsection{Security analysis}

The correctness follows immediately from the correctness of and the fact that we are using the same randomness for the Hyrax commitments across the ``show'' and ``prepare'' circuits.  

The soundness of our protocol follows from the soundness of Spartan. In particular, we can extract the full witness credential from the ``prepare'' relation.

Intuitively, zero-knowledge follows from the hiding property of the commitment scheme as well as the zero knowledge property of the Spartan zkSNARK proving system;
For proof $i$, simulator can randomly sample the linked commitment $com_i$ both distributions to reuse across both proofs, both in the commitment itself and also in the IPA used to open the Hyrax commitment to $Z(r_1, \dots, r_{\log{n}})$. 
We can show that this commitment is independent of the rest of the view of the Verifier, which consists of the following:

\begin{itemize}
    \item Sumcheck polynomials $\{s'_i(X)\}_{i \in [\log n]}$ for each of the sumchecks in Spartan
    \item $\{r_i\}$ Fiat-Shamir challenges during the sumcheck
    \item Transcript from the IPA on the sumcheck relation in ZK
    \item $\{com(z_i)\}_{i in [\log n]}$ Hyrax commitment to $Z$, which involves a Pedersen commitment to each of the rows of a $\log n \times \log n$ matrix representation of $\vec{Z}$
    \item Transcript from the IPA for opening $Z(r_1, \dots, r_n)$
    \item The claimed value of $Z(r_1, \dots, r_n)$
\end{itemize}

Since we appended random pads to $\vec{Z}$ in our ZK modifiction in Section \ref{subsubsec:zk-open-Z}, the distribution of $Z(r_1, \dots, r_n)$ is random and independent of $Z$, and therefore independent of $\{m_i\}_i$.
Furthermore, $s'_i(X)$ have totally random pads on them and their distribution is independent of $Z$, and therefore independent of $\{m_i\}_i$.
Assuming the hiding property of the Pedersen commitment schemes for messages sent during an IPA, we can also use the simulators for the IPAs without changing their joint distribution with the rest of the transcript.
% i think this works: com (a, h(a)) ~ com (b, h(a)) ~ com (b, c) from hiding property that com(a) ~ com(b) for a =/= b.

Then, we can simply run the piece-wise simulators for each zkSNARK proof for circuits $C_1$ and $C_2$ to simulate the remainder of the view. 


% \begin{enumerate}
%     \item What are key techniques that will be used in our schemes?
%     \begin{itemize}
%         \item What are its inputs and outputs?
%         \item Participants and requirements of the techniques.
%     \end{itemize}
%     \item Describes the complete scheme, from preparation to verification. 
%     \begin{itemize}
%         \item The preliminaries of users when they use our system.
%         \item What is the difference between the preparation of this scheme and the traditional way?
%         \item Some questions we also answer here but just with a compact version, the detailed or proof will be shown in Security or Appendix section.
%         \begin{itemize}
%             \item Is it a dishonest majority setting or something else? 
%             \item What happens when the setting is broken?
%             \item What trust assumptions are our scheme based on?
%             \item Is this quantum-resistant? If not, is it upgradable to quantum-resistant?
%             \item The correctness, soundness, and zk.
%         \end{itemize}
%     \end{itemize}
%     \item Discuss about the compatibility of our system when upgrading from an older system.
%     \begin{enumerate}
%         \item What happens with the older system?
%         \begin{itemize}
%             \item Is it required to change the issuance process? 
%             \item Is it a breaking change or just a soft update?
%             \item What happens with the issued ID, can it be re-used, and re-issued with a new scheme?
%             % should be revoked
%             \item \textbf{It depends on the answer about the final construction. But we have some design constraints that should be followed, which are described in slide 111 of \href{https://docs.google.com/presentation/d/1C4D8zK4gAdafgIEW-2m_qDyyT39gWo0mmFYpwmA8N3M/edit?slide=id.g338a079cb64_0_31\#slide=id.g338a079cb64_0_31}{this doc}} 
%         \end{itemize} 
%         \item What about the new PID?
%         \begin{enumerate}
%             \item What is the difference between the PID of the new scheme and the current version? Is it added more data fields to the current struct?
%             \item If not, is the process from PID to scheme input straightforward (and provable?)?
% % \textbf{            \item What if the Issuer or the ID owner wants to revoke some IDs, does it require the Issuer to update frequently? -- still need to add this}            \item If the Issuer needs to update frequently, what if they are disconnected for a while? 
%             \item What is the trust assumption?
%         \end{enumerate}
%     \end{enumerate}
%     \item After show the detailed construction, we will show more detailed about why our construction should be considered with some detailed.
%     \begin{enumerate}
%         \item What are the main advantages that make zkID outperform other solutions? 
%             \item TODO: wait for benchmarks from the technical team 
%         \item What are the trade-offs if users take our advancement?
%     \end{enumerate}
% \end{enumerate}

\section{Experiments}
\label{sec:experiments}
% \begin{enumerate}
%     \item What are the exact communication, computation, and storage costs of \begin{itemize}
%         \item The PID Provider \& Verifier.
%         \item User who uses a mobile (iOS, Android) or browser?
%         \item The detailed cost of setup, proving, and verifying steps.
%     \end{itemize}
%     \item What is the minimum hardware configuration? 
% \end{enumerate}

% \textbf{We can extract the benchmark results from \href{https://hackmd.io/@clientsideproving/zkIDBenchmarks}{this doc} (it's \href{https://github.com/privacy-scaling-explorations/zkid-benchmarks}{git repo}) }

% \jbel{Still waiting 1-2 weeks for benchmarks on the POC}

% Notably, we also show the detailed comparison between our zkID and solutions from GG and Microsoft.

\section{Conclusion}
\label{sec:conclusion}

\clearpage
\section*{Contributors and Acknowledgement}
\begin{description}
	\item[Ying Tong] description
	\item[Hy Ngo]
	\item[Janabel]
\end{description}

We thanks X Y Z (Nam Zoey and everyone who said something or helped with something) for ...

\clearpage
%%%% 8. BILBIOGRAPHY %%%%
\bibliographystyle{alpha}
\bibliography{abbrev0,crypto,biblio,references}
%%%% NOTESr
% - Download abbrev3.bib and crypto.bib from https://cryptobib.di.ens.fr/
% - Use biblio.bib for additional references not in the cryptobib database.
%   If possible, take them from DBLP.

\clearpage
\section{Appendix: EUDI Annex 2 Requirements}
\label{sec:annex}

This section is devoted to a review of the EUDI ARF's Annex 2, which covers high-level requirements for the EUDI Wallet. 
The full Annex can be found \href{https://eu-digital-identity-wallet.github.io/eudi-doc-architecture-and-reference-framework/1.4.0/annexes/annex-2/annex-2-high-level-requirements/#a231-topic-1-accessing-public-and-private-online-services-with-eudi-wallet}{here}.

\addcontentsline{toc}{section}{Annex 2 – Mapping zkID to EUDI ARF Requirements}

\noindent The EUDI Annex~2 covers over fifty topics spanning cryptographic guarantees, privacy, trust infrastructure, wallet lifecycle management, and product-level usability. We cluster the requirements into four classes depending on their relationship to the zkID protocol itself:

\begin{description}
  \item[Group A] \textbf{Directly satisfied by zkID.} \\
  Requirements that are already implemented by the zkID protocol as described in Sections~\S5--\S5.3, without additional assumptions or infrastructure. These are shown in Table~A.

  \item[Group B] \textbf{Satisfied with minor extensions.} \\
  Requirements that can be met by trivial modifications or configuration changes to zkID’s proving circuits, predicates, or interface (for instance, adding a new predicate or exposing one more public input). No new trust anchor or cryptographic primitive is required. These appear in Table~B.

  \item[Group C] \textbf{Depend on external systems or operational flows.} \\
  Requirements that require the presence of registries, PKI governance, revocation lists, wallet attestation management, or other policy frameworks external to the proving layer. zkID integrates with these systems but does not replace them. These are summarized in Table~C.

  \item[Group D] \textbf{Product / UX / policy-facing requirements.} \\
  Requirements that concern wallet behaviour, user experience, accessibility, or legal presentation. These fall outside the scope of cryptographic protocol design but are compatible with zkID’s guarantees. These are listed in Table~D.
\end{description}

\medskip
\noindent
Each table records:
\begin{itemize}
  \item the corresponding Annex~2 topic and its HLRs;
  \item a short informal description of the requirement;
  \item and the relevant subsection(s) in this document where we discuss how zkID meets or relates to the requirement.
\end{itemize}

\medskip
\noindent
The following four landscape tables provide the detailed mapping for each group:

% Mapping tables 
% Group 1
\clearpage
\begin{landscape}
\small
\begin{longtable}{p{3cm} p{10cm} p{7cm}}
\caption*{Table A — Requirements directly implemented by zkID}\\
\toprule
\textbf{Annex 2 ID} &
\textbf{Requirement} &
\textbf{Where in zkID} \\
\midrule
\endfirsthead
\toprule
\textbf{Annex 2 ID} &
\textbf{Requirement} &
\textbf{Where in zkID} \\
\midrule
\endhead
\midrule
\multicolumn{3}{r}{\emph{continued on next page}}\\
\bottomrule
\endfoot
\bottomrule
\endlastfoot

\multicolumn{3}{l}{\textbf{Topic 1 — Online Identification and Authentication (OIA)}}\\

OIA\_01 &
For OIA\_01, see Interface in \S~5. &
see Interface in \S~5. \\

OIA\_02 &
For OIA\_02, see Component prepare batches in \S~5.2.1 and Component linking in \S~5.2. &
see component prepare batches in \S~5.2.1 and component linking in \S~5.2. \\

OIA\_03a &
For OIA\_03a, see Component predicate in \S~5.1 and \S~5.2.2 and Components zkSNARK wrapper in \S~5 and \S~5.1. &
see component predicate in \S~5.1 and \S~5.2.2 and zkSNARK wrapper in \S~5 and \S~5.1. \\

OIA\_03b &
For OIA\_03b, see Prepare relation in \S~5.2 and Show relation in \S~5.2. &
see Prepare relation in \S~5.2 and Show relation in \S~5.2. \\

OIA\_03c &
For OIA\_03c, see Prepare relation in \S~5.2 and Component predicate in \S~5.1 and \S~5.2.2. &
see Prepare relation in \S~5.2 and component predicate in \S~5.1 and \S~5.2.2. \\

OIA\_04 &
For OIA\_04, see Component predicate in \S~5.1 and \S~5.2.2 and Show relation in \S~5.2. &
see component predicate in \S~5.1 and \S~5.2.2 and Show relation in \S~5.2. \\

OIA\_05 &
For OIA\_05, see Component predicate in \S~5.1 and \S~5.2.2 and Security notes in \S~6. &
see component predicate in \S~5.1 and \S~5.2.2 and Security notes in \S~3. \\

OIA\_06 &
For OIA\_06, see Component predicate in \S~5.1 and \S~5.2.2. &
see component predicate in \S~5.1 and \S~5.2.2. \\

OIA\_07 &
For OIA\_07, see Component prepare batches in \S~5.2.1, Components zkSNARK wrapper in \S~5 and \S~5.1, and Component predicate in \S~5.1 and \S~5.2.2. &
see component prepare batches in \S~5.2.1, zkSNARK wrapper in \S~5 and \S~5.1, and component predicate in \S~5.1 and \S~5.2.2. \\

OIA\_08 &
For OIA\_08, see discussion in Security notes in \S~6 and Components zkSNARK wrapper in \S~5 and \S~5.1. &
see Security notes in \S~3 and zkSNARK wrapper in \S~5 and \S~5.1. \\

OIA\_09 &
For OIA\_09, see discussion in Security notes in \S~6 and Components zkSNARK wrapper in \S~5 and \S~5.1. &
see Security notes in \S~3 and zkSNARK wrapper in \S~5 and \S~5.1. \\

OIA\_10 &
For OIA\_10, see Component predicate in \S~5.1 and \S~5.2.2 and Show relation in \S~5.2. &
see component predicate in \S~5.1 and \S~5.2.2 and Show relation in \S~5.2. \\

OIA\_11 &
For OIA\_11, see Component predicate in \S~5.1 and \S~5.2.2 and Show relation in \S~5.2. &
see component predicate in \S~5.1 and \S~5.2.2 and Show relation in \S~5.2. \\

OIA\_12 &
For OIA\_12, see Prepare relation in \S~5.2.1 and discussion in Security notes in \S~6. &
see Prepare relation in \S~5.2.1 and Security notes in \S~3. \\

OIA\_13 &
For OIA\_13, see Prepare relation in \S~5.2.1 and discussion in Security notes in \S~6. &
see Prepare relation in \S~5.2.1 and Security notes in \S~3. \\

OIA\_14 &
For OIA\_14, see Prepare relation in \S~5.2.1 and discussion in Security notes in \S~6. &
see Prepare relation in \S~5.2.1 and Security notes in \S~3. \\

OIA\_15 &
For OIA\_15, see Prepare relation in \S~5.2.1 and discussion in Security notes in \S~6. &
see Prepare relation in \S~5.2.1 and Security notes in \S~3. \\

OIA\_16 &
For OIA\_16, see Security notes in \S~6 and Component predicate in \S~5.1 and \S~5.2.2. &
see Security notes in \S~3 and component predicate in \S~5.1 and \S~5.2.2. \\[1em]

\multicolumn{3}{l}{\textbf{Topic 10 — Issuance and Credential Handling (ISSU)}}\\

ISSU\_02 &
For ISSU\_02, see components SD-JWT/mDL wrapper in \S~5 and \S~5.1 and Components zkSNARK wrapper in \S~5 and \S~5.1. &
see SD-JWT/mDL wrapper in \S~5 and \S~5.1 and zkSNARK wrapper in \S~5 and \S~5.1. \\

ISSU\_07 &
For ISSU\_07, see components Prepare relation in \S~5.2.1 and prepareCommit in \S~5.2.1. &
see Prepare relation in \S~5.2.1 and \texttt{prepareCommit} in \S~5.2.1. \\

ISSU\_08 &
For ISSU\_08, see components Prepare relation in \S~5.2.1 and prepareCommit in \S~5.2.1. &
see Prepare relation in \S~5.2.1 and \texttt{prepareCommit} in \S~5.2.1. \\

ISSU\_09 &
For ISSU\_09, see components Prover's side discussion in \S~6 and Security notes in \S~6. &
see Prover-side discussion in \S~3 and Security notes in \S~3. \\

ISSU\_10 &
For ISSU\_10, see components Prepare relation in \S~5.2.1 and Prepare relation in \S~5.2. &
see Prepare relation in \S~5.2.1 and Prepare relation in \S~5.2. \\

ISSU\_12 &
For ISSU\_12, see components SD-JWT/mDL wrapper in \S~5 and \S~5.1 and Components zkSNARK wrapper in \S~5 and \S~5.1. &
see SD-JWT/mDL wrapper in \S~5 and \S~5.1 and zkSNARK wrapper in \S~5 and \S~5.1. \\

ISSU\_12a &
For ISSU\_12a, see components SD-JWT/mDL wrapper in \S~5 and \S~5.1 and Proof interface in \S~5.2. &
see SD-JWT/mDL wrapper in \S~5 and \S~5.1 and Proof interface in \S~5.2. \\

ISSU\_16 &
For ISSU\_16, see components SD-JWT/mDL wrapper in \S~5 and \S~5.1 and Interface in \S~5. &
see SD-JWT/mDL wrapper in \S~5 and \S~5.1 and Interface in \S~5. \\

ISSU\_27 &
For ISSU\_27, see components Show relation in \S~5.2.2 and Show relation in \S~5.2. &
see Show relation in \S~5.2.2 and Show relation in \S~5.2. \\

ISSU\_33 &
For ISSU\_33, see components SD-JWT/mDL wrapper in \S~5 and \S~5.1 and Component commitment in \S~5 and \S~5.2.1. &
see SD-JWT/mDL wrapper in \S~5 and \S~5.1 and component commitment in \S~5 and \S~5.2.1. \\

ISSU\_33a &
For ISSU\_33a, see components SD-JWT/mDL wrapper in \S~5 and \S~5.1 and Component commitment in \S~5 and \S~5.2.1. &
see SD-JWT/mDL wrapper in \S~5 and \S~5.1 and component commitment in \S~5 and \S~5.2.1. \\

ISSU\_33b &
For ISSU\_33b, see components prepareCommit in \S~5.2.1 and SD-JWT/mDL wrapper in \S~5 and \S~5.1. &
see \texttt{prepareCommit} in \S~5.2.1 and SD-JWT/mDL wrapper in \S~5 and \S~5.1. \\

ISSU\_35a &
For ISSU\_35a, see components prepareCommit in \S~5.2.1 and Prepare relation in \S~5.2. &
see \texttt{prepareCommit} in \S~5.2.1 and Prepare relation in \S~5.2. \\

ISSU\_37 &
For ISSU\_37, see components Component prepare batches in \S~5.2.1 and prepareBatch in \S~5.2.1. &
see component prepare batches in \S~5.2.1 and \texttt{prepareBatch} in \S~5.2.1. \\

ISSU\_37a &
For ISSU\_37a, see components prepareBatch in \S~5.2.1 and Proof interface in \S~5.2. &
see \texttt{prepareBatch} in \S~5.2.1 and Proof interface in \S~5.2. \\

ISSU\_38 &
For ISSU\_38, see components Show relation in \S~5.2.2 and Show relation in \S~5.2. &
see Show relation in \S~5.2.2 and Show relation in \S~5.2. \\

ISSU\_39 &
For ISSU\_39, see components prepareBatch in \S~5.2.1 and Prepare relation in \S~5.2. &
see \texttt{prepareBatch} in \S~5.2.1 and Prepare relation in \S~5.2. \\

ISSU\_41a &
For ISSU\_41a, see components Show relation in \S~5.2.2 and Show relation in \S~5.2. &
see Show relation in \S~5.2.2 and Show relation in \S~5.2. \\

ISSU\_41b &
For ISSU\_41b, see components Component prepare batches in \S~5.2.1 and Show relation in \S~5.2. &
see component prepare batches in \S~5.2.1 and Show relation in \S~5.2. \\

ISSU\_41c &
For ISSU\_41c, see components Show relation in \S~5.2.2 and Show relation in \S~5.2. &
see Show relation in \S~5.2.2 and Show relation in \S~5.2. \\

ISSU\_44 &
For ISSU\_44, see components Component prepare batches in \S~5.2.1 and Show relation in \S~5.2. &
see component prepare batches in \S~5.2.1 and Show relation in \S~5.2. \\

ISSU\_58 &
For ISSU\_58, see components Show relation in \S~5.2.2 and Show relation in \S~5.2. &
see Show relation in \S~5.2.2 and Show relation in \S~5.2. \\

ISSU\_59 &
For ISSU\_59, see components Show relation in \S~5.2.2 and Show relation in \S~5.2. &
see Show relation in \S~5.2.2 and Show relation in \S~5.2. \\

ISSU\_60 &
For ISSU\_60, see components Component predicate in \S~5.1 and \S~5.2.2 and Proof interface in \S~5.2. &
see component predicate in \S~5.1 and \S~5.2.2 and Proof interface in \S~5.2. \\

ISSU\_61 &
For ISSU\_61, see components Component predicate in \S~5.1 and \S~5.2.2 and Proof interface in \S~5.2. &
see component predicate in \S~5.1 and \S~5.2.2 and Proof interface in \S~5.2. \\

ISSU\_62 &
For ISSU\_62, see components SD-JWT/mDL wrapper in \S~5 and \S~5.1 and Components zkSNARK wrapper in \S~5 and \S~5.1. &
see SD-JWT/mDL wrapper in \S~5 and \S~5.1 and zkSNARK wrapper in \S~5 and \S~5.1. \\

ISSU\_63 &
For ISSU\_63, see components prepareCommit in \S~5.2.1 and SD-JWT/mDL wrapper in \S~5 and \S~5.1. &
see \texttt{prepareCommit} in \S~5.2.1 and SD-JWT/mDL wrapper in \S~5 and \S~5.1. \\

ISSU\_64 &
For ISSU\_64, see components Proof interface in \S~5.2 and Proof interface in \S~5.2. &
see Proof interface in \S~5.2. \\[1em]

\multicolumn{3}{l}{\textbf{Topic 23 — PID and (Q)EAA issuance}}\\

Topic 23 &
23 PID issuance and (Q)EAA issuance. No HLRs, see Topic 10. &
covered by Topic 10 components in \S~5, \S~5.1, \S~5.2.1, \S~5.2.2. \\[1em]

\multicolumn{3}{l}{\textbf{Topic 47 — Protocols and interfaces for PID and (Q)EAA issuance}}\\

Topic 47 &
47 Protocols and interfaces for PID and (Q)EAA issuance and (non-)qualified. No HLRs, see Topic 10, 23. &
covered by Topic 10 / Topic 23 components in \S~5, \S~5.1, \S~5.2.1, \S~5.2.2. \\[1em]

\multicolumn{3}{l}{\textbf{Topic 53 — Zero-Knowledge Proofs (ZKP)}}\\

ZKP\_01 &
For ZKP\_01, see components Component predicate in \S~5.1 and \S~5.2.2 and Security notes in \S~6. &
see component predicate in \S~5.1 and \S~5.2.2 and Security notes in \S~3. \\

ZKP\_02 &
For ZKP\_02, see components Prepare relation in \S~5.2 and Component predicate in \S~5.1 and \S~5.2.2. &
see Prepare relation in \S~5.2 and component predicate in \S~5.1 and \S~5.2.2. \\

ZKP\_03 &
For ZKP\_03, see components Component commitment in \S~5 and \S~5.2.1 and Component linking in \S~5.2. &
see component commitment in \S~5 and \S~5.2.1 and component linking in \S~5.2. \\

ZKP\_04 &
For ZKP\_04, see components Component predicate in \S~5.1 and \S~5.2.2 and Show relation in \S~5.2. &
see component predicate in \S~5.1 and \S~5.2.2 and Show relation in \S~5.2. \\

ZKP\_05 &
For ZKP\_05, see components Prepare relation in \S~5.2 and Show relation in \S~5.2. &
see Prepare relation in \S~5.2 and Show relation in \S~5.2. \\

ZKP\_06 &
For ZKP\_06, see components Components zkSNARK wrapper in \S~5 and \S~5.1 and SD-JWT/mDL wrapper in \S~5 and \S~5.1. &
see zkSNARK wrapper in \S~5 and \S~5.1 and SD-JWT/mDL wrapper in \S~5 and \S~5.1. \\

ZKP\_07 &
For ZKP\_07, see components Prepare relation in \S~5.2.1 and Component commitment in \S~5 and \S~5.2.1 in Security notes in \S~6 and \S~5. &
see Prepare relation in \S~5.2.1 and component commitment in \S~5 and \S~5.2.1 and Security notes in \S~3. \\

ZKP\_08 &
For ZKP\_08, see components Backend modularity in \S~5.3 and Security notes in \S~6. &
see backend modularity in \S~5.3 and Security notes in \S~3. \\

ZKP\_09 &
For ZKP\_09, see components Show relation in \S~5.2.2 and Proof interface in \S~5.2. &
see Show relation in \S~5.2.2 and Proof interface in \S~5.2. \\

\end{longtable}
\end{landscape}





% Group 2
\clearpage
\begin{landscape}
\small
\begin{longtable}{p{3cm} p{10cm} p{7cm}}
\caption*{Table B — Requirements implementable by minor extension or modification of zkID components}\\
\toprule
\textbf{Annex 2 ID} &
\textbf{Requirement} &
\textbf{Coverage} \\
\midrule
\endfirsthead
\toprule
\textbf{Annex 2 ID} &
\textbf{Requirement} &
\textbf{Coverage} \\
\midrule
\endhead
\midrule
\multicolumn{3}{r}{\emph{continued on next page}}\\
\bottomrule
\endfoot
\bottomrule
\endlastfoot

\multicolumn{3}{l}{\textbf{Topic 6 — Relying Party authentication and User approval}}\\

RPA\_01 &
For RPA\_01, modify components Prepare relation in \S~5.2.1 and Prover's side discussion in \S~6. &
modify components Prepare relation in \S~5.2.1 and Prover's side discussion in \S~3. \\

RPA\_01a &
For RPA\_01a, modify components Show relation in \S~5.2 and Proof interface in \S~5.2. &
modify components Show relation in \S~5.2 and Proof interface in \S~5.2. \\

RPA\_02 &
For RPA\_02, modify components Show relation in \S~5.2 and Component predicate in \S~5.1 and \S~5.2.2. &
modify components Show relation in \S~5.2 and Component predicate in \S~5.1 and \S~5.2.2. \\

RPA\_02a &
For RPA\_02a, modify components Proof interface in \S~5.2 and Component linking in \S~5.2. &
modify components Proof interface in \S~5.2 and Component linking in \S~5.2. \\

RPA\_03 &
For RPA\_03, modify components Prepare relation in \S~5.2.1 and Show relation in \S~5.2.2. &
modify components Prepare relation in \S~5.2.1 and Show relation in \S~5.2.2. \\

RPA\_04 &
For RPA\_04, modify components Prover's side discussion in \S~6 and Security notes in \S~6. &
modify components Prover's side discussion in \S~3 and Security notes in \S~3. \\

RPA\_05 &
For RPA\_05, modify components Proof interface in \S~5.2 and Component linking in \S~5.2. &
modify components Proof interface in \S~5.2 and Component linking in \S~5.2. \\

RPA\_06 &
For RPA\_06, modify components Component predicate in \S~5.1 and \S~5.2.2 and Show relation in \S~5.2. &
modify components Component predicate in \S~5.1 and \S~5.2.2 and Show relation in \S~5.2. \\

RPA\_06a &
For RPA\_06a, modify components Show relation in \S~5.2 and Proof interface in \S~5.2. &
modify components Show relation in \S~5.2 and Proof interface in \S~5.2. \\

RPA\_07 &
For RPA\_07, modify components Show relation in \S~5.2.2 and Component predicate in \S~5.1 and \S~5.2.2. &
modify components Show relation in \S~5.2.2 and Component predicate in \S~5.1 and \S~5.2.2. \\

RPA\_07a &
For RPA\_07a, modify components Show relation in \S~5.2 and Proof interface in \S~5.2. &
modify components Show relation in \S~5.2 and Proof interface in \S~5.2. \\

RPA\_08 &
For RPA\_08, modify components Show relation in \S~5.2.2 and Show relation in \S~5.2. &
modify components Show relation in \S~5.2.2 and Show relation in \S~5.2. \\

RPA\_09 &
For RPA\_09, modify components Component predicate in \S~5.1 and \S~5.2.2 and Show relation in \S~5.2. &
modify components Component predicate in \S~5.1 and \S~5.2.2 and Show relation in \S~5.2. \\

RPA\_10 &
For RPA\_10, modify components Proof interface in \S~5.2 and Proof interface in \S~5.2. &
modify components Proof interface in \S~5.2 and Proof interface in \S~5.2. \\[1em]


\multicolumn{3}{l}{\textbf{Topic 11 — Pseudonyms}}\\

PA\_01 &
For PA\_01, modify components Component predicate in \S~5.1 and \S~5.2.2 and Prepare relation in \S~5.2. &
modify components Component predicate in \S~5.1 and \S~5.2.2 and Prepare relation in \S~5.2. \\

PA\_02 &
For PA\_02, modify components Show relation in \S~5.2 and Show relation in \S~5.2.2. &
modify components Show relation in \S~5.2 and Show relation in \S~5.2.2. \\

PA\_03 &
For PA\_03, modify components Proof interface in \S~5.2 and Proof interface in \S~5.2. &
modify components Proof interface in \S~5.2 and Proof interface in \S~5.2. \\

PA\_04 &
For PA\_04, modify components Prepare relation in \S~5.2 and Component predicate in \S~5.1 and \S~5.2.2. &
modify components Prepare relation in \S~5.2 and Component predicate in \S~5.1 and \S~5.2.2. \\

PA\_05 &
For PA\_05, modify components Proof interface in \S~5.2 and Proof interface in \S~5.2. &
modify components Proof interface in \S~5.2 and Proof interface in \S~5.2. \\

PA\_06 &
For PA\_06, modify components Show relation in \S~5.2 and Proof interface in \S~5.2. &
modify components Show relation in \S~5.2 and Proof interface in \S~5.2. \\

PA\_07 &
For PA\_07, modify components Proof interface in \S~5.2 and Proof interface in \S~5.2. &
modify components Proof interface in \S~5.2 and Proof interface in \S~5.2. \\

PA\_08 &
For PA\_08, modify components Proof interface in \S~5.2 and Proof interface in \S~5.2. &
modify components Proof interface in \S~5.2 and Proof interface in \S~5.2. \\

PA\_08a &
For PA\_08a, modify components Security notes in \S~6 and Proof interface in \S~5.2. &
modify components Security notes in \S~3 and Proof interface in \S~5.2. \\

PA\_09 &
For PA\_09, modify components Proof interface in \S~5.2 and Proof interface in \S~5.2. &
modify components Proof interface in \S~5.2 and Proof interface in \S~5.2. \\

PA\_10 &
For PA\_10, modify components Show relation in \S~5.2.2 and Security notes in \S~6. &
modify components Show relation in \S~5.2.2 and Security notes in \S~3. \\

PA\_11 &
For PA\_11, modify components Show relation in \S~5.2.2 and Security notes in \S~6. &
modify components Show relation in \S~5.2.2 and Security notes in \S~3. \\

PA\_12 &
For PA\_12, modify components Show relation in \S~5.2.2 and Show relation in \S~5.2.2. &
modify components Show relation in \S~5.2.2 and Show relation in \S~5.2.2. \\

PA\_13 &
For PA\_13, modify components Show relation in \S~5.2.2 and Proof interface in \S~5.2. &
modify components Show relation in \S~5.2.2 and Proof interface in \S~5.2. \\

PA\_14 &
For PA\_14, modify components Show relation in \S~5.2.2 and Security notes in \S~6. &
modify components Show relation in \S~5.2.2 and Security notes in \S~3. \\

PA\_15 &
For PA\_15, modify components Prepare relation in \S~5.2.1 and Security notes in \S~6. &
modify components Prepare relation in \S~5.2.1 and Security notes in \S~3. \\

PA\_16 &
For PA\_16, modify components Prepare relation in \S~5.2.1 and Component predicate in \S~5.1 and \S~5.2.2. &
modify components Prepare relation in \S~5.2.1 and Component predicate in \S~5.1 and \S~5.2.2. \\

PA\_17 &
For PA\_17, modify components Component prepare batches in \S~5.2.1 and Prepare relation in \S~5.2.1. &
modify components Component prepare batches in \S~5.2.1 and Prepare relation in \S~5.2.1. \\

PA\_18 &
For PA\_18, modify components prepareCommit in \S~5.2.1 and Security notes in \S~6. &
modify components \texttt{prepareCommit} in \S~5.2.1 and Security notes in \S~3. \\

PA\_19 &
For PA\_19, modify components Proof interface in \S~5.2 and Security notes in \S~6. &
modify components Proof interface in \S~5.2 and Security notes in \S~3. \\[1em]


\multicolumn{3}{l}{\textbf{Topic 17 — Identity matching}}\\

No HLRs &
Enable users to access existing online accounts or log in to cross-border public sector services using their PID via identity matching, even if PID attribute values do not exactly match those in the accounts. &
modify components Prepare relation in in \S~5.2.1 and Security notes in \S~3. \\[1em]

\multicolumn{3}{l}{...}\\\\[-0.5em]


\multicolumn{3}{l}{\textbf{Topic 18 — Combined presentations of attributes}}\\

ACP\_01 &
For ACP\_01, modify components Component predicate in \S~5.1 and \S~5.2.2 and Component linking in \S~5.2. &
modify components Component predicate in \S~5.1 and \S~5.2.2 and Component linking in \S~5.2. \\

ACP\_02 &
For ACP\_02, modify components Component commitment in \S~5 and \S~5.2.1 and Show relation in \S~5.2.2. &
modify components Component commitment in \S~5 and \S~5.2.1 and Show relation in \S~5.2.2. \\

ACP\_03 &
For ACP\_03, modify components Prepare relation in \S~5.2 and Show relation in \S~5.2. &
modify components Prepare relation in \S~5.2 and Show relation in \S~5.2. \\

ACP\_04 &
For ACP\_04, modify components Proof interface in \S~5.2 and Proof interface in \S~5.2. &
modify components Proof interface in \S~5.2 and Proof interface in \S~5.2. \\

ACP\_05 &
For ACP\_05, modify components Component commitment in \S~5 and Component predicate in \S~5.1 and \S~5.2.2. &
modify components Component commitment in \S~5 and Component predicate in \S~5.1 and \S~5.2.2. \\

ACP\_06 &
For ACP\_06, modify components prepareCommit in \S~5.2.1 and prepareBatch in \S~5.2.1. &
modify components \texttt{prepareCommit} in \S~5.2.1 and \texttt{prepareBatch} in \S~5.2.1. \\

ACP\_07 &
For ACP\_07, modify components Prepare relation in \S~5.2.1 and Component prepare batches in \S~5.2.1. &
modify components Prepare relation in \S~5.2.1 and Component prepare batches in \S~5.2.1. \\[1em]


\multicolumn{3}{l}{\textbf{Topic 20 — Strong User authentication for electronic payments}}\\

SUA\_01 &
For SUA\_01, modify components Show relation in \S~5.2.2 and Show relation in \S~5.2. &
modify components Show relation in \S~5.2.2 and Show relation in \S~5.2. \\

SUA\_02 &
For SUA\_02, modify components Component predicate in \S~5.1 and \S~5.2.2 and Show relation in \S~5.2. &
modify components Component predicate in \S~5.1 and \S~5.2.2 and Show relation in \S~5.2. \\

SUA\_03 &
For SUA\_03, modify components Show relation in \S~5.2 and Prepare relation in \S~5.2. &
modify components Show relation in \S~5.2 and Prepare relation in \S~5.2. \\

SUA\_04 &
For SUA\_04, modify components Show relation in \S~5.2.2 and Component predicate in \S~5.1 and \S~5.2.2. &
modify components Show relation in \S~5.2.2 and Component predicate in \S~5.1 and \S~5.2.2. \\

SUA\_05 &
For SUA\_05, modify components Security notes in \S~6 and Show relation in \S~5.2. &
modify components Security notes in \S~3 and Show relation in \S~5.2. \\

SUA\_06 &
For SUA\_06, modify components Component predicate in \S~5.1 and \S~5.2.2 and Proof interface in \S~5.2. &
modify components Component predicate in \S~5.1 and \S~5.2.2 and Proof interface in \S~5.2. \\[1em]


\multicolumn{3}{l}{\textbf{Topic 43 — Embedded disclosure policies}}\\

EDP\_01 &
For EDP\_01, modify components Component commitment in \S~5 and \S~5.2.1 and prepareCommit in \S~5.2.1. &
modify components Component commitment in \S~5 and \S~5.2.1 and \texttt{prepareCommit} in \S~5.2.1. \\

EDP\_02 &
For EDP\_02, modify components Component predicate in \S~5.1 and \S~5.2.2 and Proof interface in \S~5.2. &
modify components Component predicate in \S~5.1 and \S~5.2.2 and Proof interface in \S~5.2. \\

EDP\_03 &
For EDP\_03, modify components Prover's side discussion in \S~6 and Security notes in \S~6. &
modify components Prover's side discussion in \S~3 and Security notes in \S~3. \\

EDP\_05 &
For EDP\_05, modify components Proof interface in \S~5.2 and Proof interface in \S~5.2. &
modify components Proof interface in \S~5.2 and Proof interface in \S~5.2. \\

EDP\_06 &
For EDP\_06, modify components Component predicate in \S~5.1 and \S~5.2.2 and Show relation in \S~5.2. &
modify components Component predicate in \S~5.1 and \S~5.2.2 and Show relation in \S~5.2. \\

EDP\_07 &
For EDP\_07, modify components Component predicate in \S~5.1 and \S~5.2.2 and Show relation in \S~5.2. &
modify components Component predicate in \S~5.1 and \S~5.2.2 and Show relation in \S~5.2. \\

EDP\_09 &
For EDP\_09, modify components prepareCommit in \S~5.2.1 and SD-JWT/mDL wrapper in \S~5 and \S~5.1. &
modify components \texttt{prepareCommit} in \S~5.2.1 and SD-JWT/mDL wrapper in \S~5 and \S~5.1. \\

EDP\_10 &
For EDP\_10, modify components prepareCommit in \S~5.2.1 and Proof interface in \S~5.2. &
modify components \texttt{prepareCommit} in \S~5.2.1 and Proof interface in \S~5.2. \\

EDP\_11 &
For EDP\_11, modify components Security notes in \S~6 and Prepare relation in \S~5.2. &
modify components Security notes in \S~3 and Prepare relation in \S~5.2. \\[1em]


\multicolumn{3}{l}{\textbf{Topic 51 — PID or attestation deletion}}\\

PAD\_01 &
For PAD\_01, modify components Proof interface in \S~5.2 and Proof interface in \S~5.2. &
modify components Proof interface in \S~5.2 and Proof interface in \S~5.2. \\

PAD\_02 &
For PAD\_02, modify components prepareCommit in \S~5.2.1 and SD-JWT/mDL wrapper in \S~5 and \S~5.1. &
modify components \texttt{prepareCommit} in \S~5.2.1 and SD-JWT/mDL wrapper in \S~5 and \S~5.1. \\

PAD\_03 &
For PAD\_03, modify components Component predicate in \S~5.1 and \S~5.2.2 and Show relation in \S~5.2. &
modify components Component predicate in \S~5.1 and \S~5.2.2 and Show relation in \S~5.2. \\

PAD\_04 &
For PAD\_04, modify components Show relation in \S~5.2.2 and Show relation in \S~5.2. &
modify components Show relation in \S~5.2.2 and Show relation in \S~5.2. \\

PAD\_05 &
For PAD\_05, modify components Security notes in \S~6 and Component commitment in \S~5 and \S~5.2.1. &
modify components Security notes in \S~3 and Component commitment in \S~5 and \S~5.2.1. \\

PAD\_06 &
For PAD\_06, modify components Component prepare batches in \S~5.2.1 and Show relation in \S~5.2. &
modify components Component prepare batches in \S~5.2.1 and Show relation in \S~5.2. \\[1em]


\multicolumn{3}{l}{\textbf{Topic 52 — Relying Party intermediaries}}\\

\multicolumn{3}{l}{...} \\

\end{longtable}
\end{landscape}



% Group 3
\clearpage
\begin{landscape}
\small
\begin{longtable}{p{3cm} p{10cm} p{7cm}}
\caption*{Table C — Requirements requiring integration with external systems or protocol adaptations}\\
\toprule
\textbf{Annex 2 Topic / ID} &
\textbf{Requirement} &
\textbf{Coverage} \\
\midrule
\endfirsthead
\toprule
\textbf{Annex 2 Topic / ID} &
\textbf{Requirement} &
\textbf{Coverage} \\
\midrule
\endhead
\midrule
\multicolumn{3}{r}{\emph{continued on next page}}\\
\bottomrule
\endfoot
\bottomrule
\endlastfoot

\multicolumn{3}{l}{\textbf{Topic 2 — Mobile Driving Licence (mDL) within the EUDI Wallet ecosystem}}\\

Topic 2 &
 &
\\

\multicolumn{3}{l}{\textbf{Topic 3 — PID Rulebook}}\\

Topic 3 &
 &
\\

\multicolumn{3}{l}{\textbf{Topic 4 — mDL Rulebook}}\\

Topic 4 &
 &
\\

\multicolumn{3}{l}{\textbf{Topic 7 — Attestation revocation and revocation checking}}\\

Topic 7 &
 &
\\

\multicolumn{3}{l}{\textbf{Topic 9 — Wallet Instance Attestation / Wallet Unit Attestation}}\\

Topic 9 &
 &
\\

\multicolumn{3}{l}{\textbf{Topic 16 — Signing documents with a Wallet Unit}}\\

Topic 16 &
 &
\\

\multicolumn{3}{l}{\textbf{Topic 24 — User identification in proximity scenarios}}\\

Topic 24 &
 &
\\

\multicolumn{3}{l}{\textbf{Topic 25 — Unified definition and controlled vocabularies for attributes}}\\

Topic 25 &
 &
\\

\multicolumn{3}{l}{\textbf{Topic 26 — Catalogue of attestations}}\\

Topic 26 &
 &
\\

\multicolumn{3}{l}{\textbf{Topic 27 — Registration of PID Providers, Providers of QEAAs, PuB-EAAs, and non-qualified}}\\

Topic 27 &
 &
\\

\multicolumn{3}{l}{\textbf{Topic 30 — Interaction between Wallet Units}}\\

Topic 30 &
 &
\\

\multicolumn{3}{l}{\textbf{Topic 31 — Notification and publication of PID Provider / Wallet Provider / Attestation Provider trust status}}\\

Topic 31 &
 &
\\

\multicolumn{3}{l}{\textbf{Topic 33 — Wallet Unit backup and restore}}\\

Topic 33 &
 &
\\

\multicolumn{3}{l}{\textbf{Topic 35 — PID issuance service blueprint}}\\

Topic 35 &
 &
\\

\multicolumn{3}{l}{\textbf{Topic 37 — QES / Remote Signing — Technical Requirements}}\\

Topic 37 &
 &
\\

\multicolumn{3}{l}{\textbf{Topic 38 — Wallet Unit revocation}}\\

Topic 38 &
 &
\\

\multicolumn{3}{l}{\textbf{Topic 39 — Wallet-to-wallet technical topic}}\\

Topic 39 &
 &
\\

\multicolumn{3}{l}{\textbf{Topic 40 — Wallet Instance installation / activation / management}}\\

Topic 40 &
 &
\\

\multicolumn{3}{l}{\textbf{Topic 44 — Registration certificates for PID Providers, QEAAs, PuB-EAAs}}\\

Topic 44 &
 &
\\

\multicolumn{3}{l}{\textbf{Topic 48 — Blueprint for requesting data deletion to Relying Parties}}\\

Topic 48 &
 &
\\

\end{longtable}
\end{landscape}



% Group 4
\clearpage
\begin{landscape}
\small
\begin{longtable}{p{3cm} p{10cm} p{7cm}}
\caption*{Table D — Requirements dependent on product, user interface, or user-experience considerations}\\
\toprule
\textbf{Annex 2 Topic / ID} & \textbf{Requirement} & \textbf{Coverage}\\
\midrule
\endfirsthead
\toprule
\textbf{Annex 2 Topic / ID} & \textbf{Requirement} & \textbf{Coverage}\\
\midrule
\endhead
\midrule
\multicolumn{3}{r}{\emph{continued on next page}}\\
\bottomrule
\endfoot
\bottomrule
\endlastfoot

Topics 5, 8, 13, 14, 15, 21, 22, 36 &
There are no HLRs for this Topic. & \\

Topics 12, 32, 41, 45, 46 &
 & Refers to the attestation rulebook \\

Topics 24, 30 &
 & Refers to wallet instance requirements \\

Topic 28 &
 & Refers to PID rulebook \\

Topic 29 &
 & Refers to natural person PID \\

Topic 33 &
 & Refers to functional requirements of back up and restore function of wallet instance \\

Topic 42 &
... & ... \\

Topic 50 &
& Refers to compliance requirements of the wallet provider and wallet instance \\

Topic 54 &
... & ... \\

\end{longtable}
\end{landscape}


\end{document}


It is widely recognized that digital identity systems must deliver security, perform reliably on everyday hardware, and preserve user privacy across multiple sessions, and it has also been observed that these requirements become most challenging when deployments move beyond controlled environments to heterogeneous devices, fluctuating networks, and varied implementations~\cite{PoPETS:KruPaiRujKan24}. Within the European Union’s European Digital Identity (EUDI) Architecture \& Reference Framework~\cite{EU:EUDI23}, these aims are stated explicitly through selective disclosure and unlinkability. Achieving security, reliability, and privacy in digital identity systems requires not only robust cryptographic choices but also practical integration into real-world environments. We consider a typical deployment: wallets run on commodity smartphones, verifiers operate within standard web infrastructures, and network conditions vary widely \cite{EU:EUDI23}. In such settings, stable identifiers or protocol patterns can enable unintended correlation of user interactions, compromising privacy. Effective designs must ensure predictable performance under load, minimal resource demands on devices, and low server-side verification costs, while avoiding non-standard cryptographic assumptions that hinder adoption \cite{NIST:Grassi17}. Our zkID system addresses these challenges by prioritizing unlinkability and deployability, as we explore in subsequent sections.

Security and privacy are distinct requirements in digital identity systems: while authenticity and integrity can be ensured, linkability often persists, compromising user privacy~\cite{PoPETS:KruPaiRujKan24,JC:FeiFiaSha88}. We work with the usual roles—issuer, holder, verifier—and with two flows: issuance and presentation. Correlation can arise from deterministic disclosure structures, from identifiers that remain stable (intended or incidental), and from protocol- or transport-level regularities across sessions. Unless policy authorizes correlation, two valid presentations should not be tied back to the same credential; this is the privacy target used here.

Standardized formats such as Selective Disclosure JWT (SD-JWT)~ \cite{IETF:FetYasCam25} and ISO mobile Driving Licence (mDL)~\cite{ISO:18013-5} were designed to minimize disclosed attributes and do so effectively. Even so, disclosure encodings and patterns can be stable enough for cooperating verifiers to correlate presentations over time. Pairing-based credentials (e.g., BBS+\cite{C:BCTV14}) address unlinkability at the credential layer; however, reliance on pairing-friendly curves complicates certification paths and leaves post-quantum migration unsettled in ecosystems that prioritize NIST-standardized primitives~\cite{EC:TesZhu23b}.

A third direction wraps standardized credentials in general-purpose zero-knowledge proofs. This can suppress correlating structure, yet it reintroduces a familiar fork: heavier verification without a common reference string, or a universal setup whose governance is difficult to reconcile with open, multi-party standardization. The question that follows is straightforward to state, although less simple to answer:
\begin{center}
    \textit{Can zero-knowledge be layered over existing, standardized credentials so that unlinkability holds against cooperating verifiers, while practical latency and verifier cost are preserved, and without introducing a universal trusted setup or dependence on non-standard curves?}
\end{center}

For a solution to be acceptable, it should satisfy four requirements. First, deployability: it must interoperate with mDoc/JWT data structures and existing PKI (ECDSA or RSA) without changes to issuer processes or secure elements. Second, performance: proof generation and verification must remain within mobile and web latency budgets, and expensive steps (e.g., issuer-signature checks) should be cached or batched when possible. Third, unlinkability: distinct presentations of the same credential must be non-correlatable under the stated threat model, including colluding verifiers. Finally, transparency and post-quantum agility: no trusted setup and no pairing-friendly curves; rely on standard assumptions (e.g., discrete logarithm) and allow PQ replacements without redesigning the protocol.

\subsection{Our zkID Construction}

Motivated by these requirements, we propose \emph{zkID}, a modular zk-SNARK wrapper aimed at real-world identity workflows. At a high level, the design follows a simple rule: pre-process once, reuse later. Proof work is split into two phases:
\begin{itemize}
    \item \textbf{Prepare.} Run infrequently. The issuer’s ECDSA signature is verified, credential attributes are parsed from SD-JWT (or related formats), and SHA-256 hashes are computed over intended disclosures. The results are committed with Hyrax vector commitments and can be cached~\cite{cryptoeprint:2017/1132}.
    \item \textbf{Show.} Run per presentation: selected attributes or predicates are disclosed, a nonce-bound signature from the holder’s secure device is validated, and equality is proved against cached commitments from the \emph{Prepare} phase.
\end{itemize}

Our construction leverages Spartan interactive-oracle proofs over the Tom256 elliptic curve, aligning with ECDSA infrastructures (e.g., NIST P-256) and using Hyrax commitments for efficient linking \cite{C:Setty20,cryptoeprint:2017/1132}. Our evaluation considers a standard adversary model with malicious holders, colluding verifiers, and honest-but-curious issuers, deferring quantum adversaries and side-channel attacks to future work. To highlight differences in zkID’s approach, we review related privacy-preserving credential systems, comparing their strategies for unlinkability and deployability with our solution \cite{cryptoeprint:2024/2010,cryptoeprint:2024/2013}.

\subsection{Related Work}

\paragraph{Anonymous Credentials from ECDSA.}
Matteo Frigo and Abhi Shelat (Google) propose an anonymous credential scheme for legacy-deployed Elliptic Curve Digital Signature Algorithm (ECDSA), prioritizing large-scale deployability~\cite{cryptoeprint:2024/2010}. Their design avoids changes to issuer processes, mobile devices (including secure elements), and non-standard cryptographic assumptions, unlike prior schemes like BBS+. This addresses reluctance of organizations to update infrastructure that supports only RSA or ECDSA.

To overcome the bottleneck of ZK proofs for ECDSA with non-NTT-friendly curves like P-256, they employ a ZK proof system built around sum-check and the Ligero argument system~\cite{CCS:AHIV17}. Efficiency is achieved through specialized ECDSA circuits and efficient Reed–Solomon encoding. They focus on reducing prover time and energy use for mobile devices. A witness consistency protocol is introduced for optimal performance when combining ECDSA proofs (over prime fields like $\mathbb{F}_{p}$ with $p\approx 2^{256}$) with SHA-256 hashes (efficient over binary fields like $\mathrm{GF}(2^k)$), ensuring common witness values remain consistent across different field operations~\cite[\S3.4]{cryptoeprint:2024/2010}.

Their system reports ECDSA proofs generated in 60\,ms~\cite[\S5.3]{cryptoeprint:2024/2010} and a zero-knowledge proof for the ISO/IEC 18013-5 mDL presentation flow generated in 1.2\,s on mobile devices~\cite[\S6.2]{cryptoeprint:2024/2010}. Notably, the scheme does not require any trusted setup, which is considered impractical for anonymous credential systems with many issuers and relying parties. Device binding follows the mDL live-challenge pattern. The main trade-off is a comparatively higher verifier workload and larger proof sizes than succinct CRS-based SNARKs.

\paragraph{Crescent Credentials.} 
Christian Paquin, Guru-Vamsi Policharla, and Greg Zaverucha introduce Crescent, designed to upgrade privacy features of existing credentials like JWTs and mDLs~\cite{cryptoeprint:2024/2013}. It achieves selective disclosure and unlinkability without requiring new parties or changes to existing issuance infrastructure.

The system uses a two-phase workflow:
\begin{itemize}
  \item Prepare: A computationally intensive, offline pre-processing step run once per credential. It uses Groth16 to prove that the signed credential decodes and parses to attributes, outputting a Pedersen vector commitment. Reported timings: approximately 27\,s for a 2\,KB JWT and 140\,s for an mDL, with universal-setup parameters ranging from 661\,MB to 1.1\,GB~\cite[\S4]{cryptoeprint:2024/2013}.
  \item Show: Subsequent presentation proofs are much faster: approximately 22\,ms for a JWT and 41.2\,ms for an mDL, combining re-randomized Groth16 proofs with small $\Sigma$-proofs over committed attributes. Proofs are around 1\,KB, often shorter than the input credential~\cite[\S4]{cryptoeprint:2024/2013}.
\end{itemize}

Crescent’s modular design, using committed attribute values, allows integration of sub-provers for advanced predicates (e.g., range proofs or credential linking). Its primary trade-off is reliance on a trusted setup for Groth16. Crescent also supports device-bound credentials via a sub-prover that demonstrates knowledge of a signature verifying under a committed public key, linked to the main Groth16 proof.

\paragraph{zkID.}

zkID is a modular, transparent (no-CRS) zero-knowledge wrapper for standardized credentials, prioritizing unlinkability and deployability. We adopt the same two-phase workflow as Crescent but avoid a universal setup, and we retain ECDSA compatibility as in Google’s design while moving issuer verification to the offline phase to reduce online verifier work. Key distinctions from prior work include:
\begin{itemize}
  \item Transparency and post-quantum agility: no trusted setup; standard discrete-log assumptions; architecture designed to swap in PQ vector commitments in future instantiations.
  \item Deployment compatibility: interoperates with mDL/SD-JWT and existing PKI (ECDSA or RSA) without changes to issuer processes or secure elements.
  \item Reduced online burden: compared to Google’s approach, issuer-signature verification and parsing move to the offline Prepare phase, reducing online compute.
  \item Avoidance of pairing-based cryptography: unlike pairing-based schemes, zkID avoids pairing-friendly curves.
  \item Verifier cost: reduces verifier work relative to prior ECDSA-based designs under our constraints while preserving unlinkability under the stated threat model.
\end{itemize}

In summary, zkID shows that a modular, standards-aligned zero-knowledge wrapper over existing credential formats can be built. Compared to pairing-based schemes, it avoids dependence on pairing-friendly curves; compared to universal-setup SNARK wrappers, it avoids a trusted setup; and relative to prior ECDSA-based designs, it reduces verifier cost while preserving unlinkability under our model. The construction is transparent and PQC-agile; the current instantiation is classical and not quantum-resistant.

In our evaluation, zkID achieves practical performance on commodity hardware. For a typical credential with ..., the one-time \emph{Prepare} phase completes in ... seconds on a standard laptop (or mobile) (..., ...GB RAM), while the per-presentation \emph{Show} phase takes less than ... ms on a mid-range smartphone. Proof sizes are approximately ... KB for the full protocol, and verifier time remains below ... ms on a web server. These results demonstrate that zkID meets latency and resource targets for mobile and web deployments, while preserving unlinkability and avoiding trusted setup. Detailed benchmark results and further breakdowns will be provided in Section~\ref{sec:experiments}.

Several aspects are left out of scope. We do not provide comprehensive revocation, explicit defenses against hardware side channels, or end-to-end wallet-integration benchmarks, and a concrete instantiation of lattice-based vector commitments is deferred to future work.
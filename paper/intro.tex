A secure, efficient, and privacy-preserving digital identity infrastructure has become a central objective for governmental and commercial applications worldwide. The European Union’s recent push toward a standardized digital identity framework (the so-called European Digital Identity (EUDI) Architecture$\&$ Reference Framework) highlights this importance clearly: users must be able to selectively disclose attributes from their identity credentials without revealing unnecessary personal information, while simultaneously guaranteeing strong unlinkability across multiple interactions. However, translating these requirements into cryptographic practice remains challenging.~\cite{TODO}

Current identity credential formats, notably SD-JWT and mDL, do not provide unlinkability properties. In particular, SD-JWT presentations remain vulnerable to correlation attacks across cooperating verifiers due to their deterministic disclosure structure. Pairing-based credential alternatives, such as BBS$\#$ and BBS+, do provide partial unlinkability guarantees, but their reliance on non-standard pairing-friendly elliptic curves presents a barrier to compliance with established security certifications, and their construction currently lacks a clear migration path toward post-quantum cryptographic security. Recent proposals employing generic zk-SNARK approaches (for example, Google's anonymous credential constructions or Microsoft's Crescent scheme) similarly exhibit significant limitations: ...

Thus, the design of a modular, efficient, and openly specified zero-knowledge (ZK) wrapper for standard credential formats remains a clear and immediate research problem. Such a solution must achieve the following precise set of requirements:
\begin{itemize}
    \item Performance: The solution must enable proof generation and verification to complete comfortably within the tight latency budgets typically expected for mobile devices. To achieve this, heavy cryptographic operations such as credential parsing and issuer-signature verification should ideally be computed once and amortized over multiple presentations.
    \item Strong Unlinkability: The proof system must provide cryptographic guarantees that distinct presentation events from the same credential cannot be correlated by collaborating verifiers. Such unlinkability must hold under realistic threat models, explicitly considering malicious or colluding relying parties.
    \item No Trusted Setup and Standardized Cryptography: In zkID, we prioritize avoiding trusted setups and pairing-friendly curves, relying only on standardized cryptographic assumptions like discrete-logarithm hardness. Its modular design, with clear specifications for primitives like interactive oracle proofs and polynomial commitments, supports open standardization for interoperability in forums like IETF and W3C.
\end{itemize}

In this paper, we propose zkID, a modular zk-SNARK wrapper explicitly designed to satisfy these requirements. At a conceptual level, our design follows a simple principle: "pre-process once, reuse proofs efficiently many times." Specifically, our construction cleanly separates proof generation into two distinct computational phases:
\begin{itemize}
    \item **Prepare**: Infrequently executed, this phase involves verifying the issuer's ECDSA signature, parsing the credential attributes from the SD-JWT (or similar credential formats), and computing SHA-256 hashes over attribute disclosures. The results are committed using Hyrax vector commitments, which can then be cached for subsequent reuse.
    \item **Show**: Executed at every credential presentation, this phase selectively discloses chosen attributes or predicates, verifies a nonce-based signature generated by the holder's secure cryptographic device (WSCD), and efficiently proves equality against cached commitments from the prepare phase.
\end{itemize}

Our construction leverages Spartan’s interactive oracle proof system over the Tom256 elliptic curve group, chosen explicitly to align arithmetic operations cleanly with existing ECDSA infrastructures (e.g., NIST P-256 keys). Additionally, the commitment linking between the prepare and show phases is achieved directly via Hyrax vector commitment equality checks, removing the complexity and overhead of MAC-based or hash-based linking gadgets used in previous proposals.

We assume a standard adversary model for credential proofs: a malicious credential holder attempting forgery or replay, colluding verifiers trying to track users via proof correlation, and an honest-but-curious issuer observing interactions. Quantum adversaries and side-channel attacks on secure hardware are left for future work.

Our zkID construction relies on discrete-logarithm hardness in the Tom256 group and the knowledge-soundness of Spartan interactive oracle proofs under the random-oracle model. It achieves statistical zero-knowledge, knowledge soundness, unlinkability across presentations, selective attribute disclosure, and proof re-randomization. For post-quantum security, replacing the Hyrax commitment with lattice-based vector commitments enables a seamless transition without altering the Spartan proof structure.

We developed a research prototype with a Spartan backend over Tom256, a Hyrax commitment library, a CLI for offline batch generation. While full wallet integration is future work, this implementation demonstrates the feasibility of zkID’s efficiency and cryptographic properties.

Finally, we briefly summarise how our zkID proposal compares to other existing solutions:
\begin{itemize}
    \item BBS$\#$ and BBS+: Unlike these pairing-based credential solutions, zkID does not require pairing-friendly curves or trusted setup, and provides richer predicate expressiveness alongside a clear post-quantum migration strategy.
    \item Google’s anonymous credentials: our Hyrax-based linking is simpler and avoids additional MAC-based linking gadgets.
    \item Microsoft’s Crescent Credentials: zkID completely avoids Crescent’s universal trusted KZG setup, enabling a simpler standardization process and easier adoption in practice.
\end{itemize}

Nevertheless, we acknowledge several limitations: practical performance benchmarks and precise memory footprints can only be accurately evaluated after integration into a full-scale wallet; explicit revocation mechanisms, protection against hardware side-channel attacks, and complete post-quantum security instantiation remain clear open directions for subsequent research efforts.

People use digital credentials to prove a single fact$-$such as being over 18, holding a license, or belonging to an organization$-$without presenting an entire document. In the W3C model~\cite{w3c-vc-data-model-2}, an issuer signs a credential, a holder keeps it (typically in a wallet application), and a verifier checks a presentation—that is, data derived from one or more credentials and shown to a specific verifier for a specific session. This is the baseline flow we adopt.

A practical privacy risk arises when the same credential is shown to different services over time. If a presentation exposes a stable technical trace$-$for example, an identifier or handle that repeats$-$then services can compare logs later and infer that the sessions likely came from the same source.
We call this linkability: separate presentations that should remain independent become easy to connect. Standards documents explicitly caution against this: the W3C Verifiable Credentials Data Model states that securing mechanisms must not leak information that would enable a verifier to correlate a holder across multiple presentations~\cite{w3c-vc-data-model-2}; and NIST’s federation guidance recommends pairwise pseudonymous identifiers$-$meaning unique, opaque identifiers for each relying party$-$to ensure separation across services and reduce the risk of user tracking~\cite{nist-sp800-63c}.
Deployments typically provide two recurring capabilities: device binding (confirming that the credential is presented by the authorized wallet) and revocation/status (learning whether the credential is still valid). To support these, systems often expose stable values.
For device binding, some ecosystems tie a credential to a long-lived cryptographic key under the control of the device so a verifier can accept only “the right wallet”; if the same key reference or its proof is visible across sessions, different verifiers can match their logs~\cite{IETF:FetYasCam25}.
For status, verifiers query whether a credential was suspended or revoked; a one-to-one status reference (for example, a unique status URL per credential) lets the status provider observe which verifier asked about which holder and when~\cite{w3c-vc-data-model-2}.
These two mechanisms keep systems reliable and usable, but because they rely on fixed values, they also leave repeating patterns that make it possible to link separate credential uses.
\dots
\subsection{Related Work}

For the purpose of comparison, we first outline a reference architecture that represents what an anonymous-credential system would ideally look like if it is to integrate smoothly with current infrastructures. In this model, the issuer is treated as fixed components that continue to use their existing public-key algorithms (such as RSA or ECDSA) and standard credential formats (e.g., JWT or mDL), since it's typically difficult to change once deployed. All additional logic is placed in the user’s wallet and the verifier.
The wallet is expected to operate in two stages: an offline Prepare step, which verifies the issuer’s signature once using standard libraries, parses and normalizes credential attributes (for example, turning a date of birth into an integer age), and commits to those attributes using a binding and hiding commitment scheme (a cryptographic way to lock values so they can later be revealed or proven in restricted form); and an online Show step, which runs per presentation, where the wallet selects only the attributes or predicates required by a relying party’s policy, proves them in zero knowledge against the stored commitments, and includes a fresh device signature over the session challenge to ensure the proof is tied to the holder’s device.
A further requirement is modularity: each major function—issuer signature verification, attribute commitment, predicate proofs, and device binding—should be defined as a separate module with a clear interface. This separation makes it possible to swap the underlying proof engine (for example, using a SNARK today or a post-quantum proof system in the future) without requiring changes to parts of the system that are costly or impractical to modify. The purpose of this modular view is to act as a comparison framework: it outlines how a deployment-friendly anonymous-credential stack could be structured, making it easier to compare proposals by the modules they cover, the constraints they address, and the trade-offs they make.

\paragraph{Anonymous Credentials from ECDSA.}
Matteo Frigo and Abhi Shelat (Google) propose an anonymous credential scheme for legacy-deployed ECDSA that preserves deployability on existing infrastructures (e.g., P-256, SHA-256) without issuer or device changes and without trusted setup, addressing reluctance to upgrade infrastructures that only support RSA/ECDSA~\cite{cryptoeprint:2024/2010}. Their design tackles the bottleneck of zero-knowledge proofs over non-NTT-friendly curves by building a proof stack around sum-check and the Ligero argument~\cite{CCS:AHIV17}, introducing specialized ECDSA circuits and efficient Reed–Solomon encodings, and adding a witness-consistency mechanism to keep common witness values aligned across arithmetic over $\mathbb{F}_{p}$ (ECDSA) and $\mathrm{GF}(2^k)$ (SHA-256). As reported, the system generates a proof for a single ECDSA signature in 60 ms and a zero-knowledge proof for the ISO/IEC 18013-5 mDL presentation flow in 1.2 s on mobile devices~\cite[\S5.3,\S6.2]{cryptoeprint:2024/2010}. Device binding follows the mDL live-challenge pattern. The main trade-off is comparatively larger proofs and higher verifier workload than succinct CRS-based SNARKs.

\paragraph{Crescent Credentials.}
Christian Paquin, Guru-Vamsi Policharla, and Greg Zaverucha introduce Crescent to upgrade privacy of existing credentials (JWTs, mDL) with selective disclosure and unlinkability without changing issuance processes~\cite{cryptoeprint:2024/2013}. Crescent achieves fast online presentations with compact proofs, while relying on a two-phase workflow:
\begin{itemize}
  \item Prepare (offline, once per credential): verify issuer signatures, decode and parse the credential into attributes, and produce a Pedersen vector commitment via a Groth16 proof. Reported timings are approximately 27 s for a 2 KB JWT and 140 s for an mDL, with universal-setup parameters of size approximately 661 MB–1.1 GB~\cite[\S4]{cryptoeprint:2024/2013}.
  \item Show (online, per presentation): re-randomize the Groth16 proof and attach a few proofs over committed attributes; typical latencies are approximately 22 ms for JWT and 41.2 ms for mDL, with proofs around 1 KB. Optional device binding increases these costs (e.g., about 315 ms Show, about 184 ms verify; proof size about 15 KB)~\cite[\S4]{cryptoeprint:2024/2013}.
\end{itemize}
Crescent exposes a modular committed-attribute interface that admits sub-provers for range checks, credential linking, and binding to session context or device by proving knowledge of a signature under a committed public key. The principal trade-offs are reliance on pairing-based Groth16 with a trusted setup and large universal parameters, and a comparatively heavy (but amortized) offline Prepare phase; in its current form Crescent is not post-quantum secure~\cite{groth2016size}.

\paragraph{zkID.}
zkID is a modular, transparent (no-CRS) zero-knowledge wrapper for standardized credentials, prioritizing unlinkability and deployability. We adopt the same two-phase workflow as Crescent but avoid a universal setup, and we retain ECDSA compatibility as in Google’s design while moving issuer verification to the offline phase to reduce online verifier work. Key distinctions from prior work include:
\begin{itemize}
  \item Transparency: no trusted setup; standard discrete-log assumptions; 
  \item Deployment compatibility: interoperates with mDL/SD-JWT and existing PKI (ECDSA or RSA) without changes to issuer processes or secure elements.
%   \item Reduced online burden: relative to prior ECDSA-based designs, issuer-signature verification and parsing move to the offline Prepare phase, reducing online compute.
  \item Avoidance of pairing-based cryptography: unlike pairing-based schemes, zkID avoids pairing-friendly curves.
  \item Verifier cost: reduces verifier work relative to prior ECDSA-based designs under our constraints while preserving unlinkability under the stated threat model.
\end{itemize}


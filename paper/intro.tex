
It is widely accepted that digital identity systems need to ensure security, work reliably on common devices, and protect user privacy over repeated uses. These goals become much harder to achieve when systems are used in everyday settings, handling different types of platforms, various implementations, and unreliable networks.~\cite{PoPETS:KruPaiRujKan24}. Within the European Union’s European Digital Identity (EUDI) Architecture \& Reference Framework~\cite{EU:EUDI23}, these aims are stated explicitly through selective disclosure and unlinkability. Achieving security, reliability, and privacy in digital identity systems requires not only robust cryptographic choices but also practical integration into real-world environments. We consider a typical deployment: wallets run on commodity smartphones, verifiers operate within standard web infrastructures, and network conditions vary widely \cite{EU:EUDI23}. In such settings, stable identifiers or protocol patterns can enable unintended correlation of user interactions, compromising privacy. Effective designs must ensure predictable performance under load, minimal resource demands on devices, and low server-side verification costs, while avoiding non-standard cryptographic assumptions that hinder adoption \cite{NIST:Grassi17}. The central challenge is to reconcile deployability, unlinkability, and efficiency under real-world conditions. To make this challenge precise, it is necessary to clarify the privacy requirement that underlies unlinkability.

A key distinction here is that security and privacy are separate requirements: while authenticity and integrity can be ensured, linkability often persists, compromising user privacy~\cite{PoPETS:KruPaiRujKan24,JC:FeiFiaSha88}. In standard identity workflows, the usual roles are issuer, holder, and verifier, with two basic flows: issuance and presentation. Correlation may arise from deterministic disclosure structures, from identifiers that remain stable—whether intended or incidental—or from protocol- or transport-level regularities across sessions. The privacy requirement is therefore that two valid presentations remain unlinkable, unless correlation is authorized by policy. With this requirement in mind, we now turn to existing credential formats and cryptographic approaches, and examine how well they address unlinkability in practice.

Standardized formats such as Selective Disclosure JWT (SD-JWT)~ \cite{IETF:FetYasCam25} and ISO mobile Driving Licence (mDL)~\cite{ISO:18013-5} were designed to minimize disclosed attributes and do so effectively. Even so, disclosure encodings and patterns can be stable enough for cooperating verifiers to correlate presentations over time. Pairing-based credentials (e.g., BBS+\cite{C:BCTV14}) address unlinkability at the credential layer; however, reliance on pairing-friendly curves complicates certification paths and leaves post-quantum migration unsettled in ecosystems that prioritize NIST-standardized primitives~\cite{EC:TesZhu23b}. Another direction instead wraps standardized credentials in general-purpose zero-knowledge proofs. This can suppress correlating structure, yet it reintroduces a familiar fork: heavier verification without a common reference string, or a universal setup whose governance is difficult to reconcile with open, multi-party standardization. The question that follows is straightforward to state, although less simple to answer:
\begin{center}
    \textit{Can zero-knowledge be layered over existing, standardized credentials so that unlinkability holds against cooperating verifiers, while practical latency and verifier cost are preserved, and without a universal trusted setup or dependence on non-standard curves?}
\end{center}

For a solution to be acceptable, it should satisfy four requirements. First, deployability: it must interoperate with mDL/JWT data structures and existing PKI (ECDSA or RSA) without changes to issuer processes or secure elements. Second, performance: proof generation and verification must remain within mobile and web latency budgets, and expensive steps (e.g., issuer-signature checks) should be cached or batched when possible. Third, unlinkability: distinct presentations of the same credential must be non-correlatable under the stated threat model, including colluding verifiers. Finally, transparency and post-quantum agility: no trusted setup and no pairing-friendly curves; rely on standard assumptions (e.g., discrete logarithm) and allow PQ replacements without redesigning the protocol.

\subsection{Our zkID Construction}

Building on these requirements, we propose \emph{zkID}, a modular zk-SNARK wrapper aimed at real-world identity workflows. At a high level, the design follows a simple rule: pre-process once, reuse later. Proof work is split into two phases:
\begin{itemize}
    \item \textbf{Prepare.} Run infrequently. The issuer’s ECDSA signature is verified, credential attributes are parsed from SD-JWT (or related formats), and SHA-256 hashes are computed over intended disclosures. The results are committed with Hyrax vector commitments and can be cached per credential~\cite{cryptoeprint:2017/1132}.
    \item \textbf{Show.} Run per presentation: selected attributes or predicates are disclosed, the device-bound nonce signature is validated, and equality is proved against cached commitments from the \emph{Prepare} phase.
\end{itemize}

Our construction leverages Spartan interactive-oracle proofs over the Tom256 elliptic curve, aligning with ECDSA infrastructures (e.g., NIST P-256) and using Hyrax commitments for efficient linking \cite{C:Setty20,cryptoeprint:2017/1132}. Our evaluation considers a standard adversary model with malicious holders, colluding verifiers, and honest-but-curious issuers, deferring quantum adversaries and side-channel attacks to future work. To highlight differences in zkID’s approach, we review related privacy-preserving credential systems, comparing their strategies for unlinkability and deployability with our solution \cite{cryptoeprint:2024/2010,cryptoeprint:2024/2013}.

\subsection{Related Work}

\paragraph{Anonymous Credentials from ECDSA.}
Matteo Frigo and Abhi Shelat (Google) propose an anonymous credential scheme for legacy-deployed ECDSA that preserves deployability on existing infrastructures (e.g., P-256, SHA-256) without issuer or device changes and without trusted setup, addressing reluctance to upgrade infrastructures that only support RSA/ECDSA~\cite{cryptoeprint:2024/2010}. Their design tackles the bottleneck of zero-knowledge proofs over non-NTT-friendly curves by building a proof stack around sum-check and the Ligero argument~\cite{CCS:AHIV17}, introducing specialized ECDSA circuits and efficient Reed–Solomon encodings, and adding a witness-consistency mechanism to keep common witness values aligned across arithmetic over $\mathbb{F}_{p}$ (ECDSA) and $\mathrm{GF}(2^k)$ (SHA-256). As reported, the system generates a proof for a single ECDSA signature in 60 ms and a zero-knowledge proof for the ISO/IEC 18013-5 mDL presentation flow in 1.2 s on mobile devices~\cite[\S5.3,\S6.2]{cryptoeprint:2024/2010}. Device binding follows the mDL live-challenge pattern. The main trade-off is comparatively larger proofs and higher verifier workload than succinct CRS-based SNARKs.

\paragraph{Crescent Credentials.}
Christian Paquin, Guru-Vamsi Policharla, and Greg Zaverucha introduce Crescent to upgrade privacy of existing credentials (JWTs, mDL) with selective disclosure and unlinkability without changing issuance processes~\cite{cryptoeprint:2024/2013}. Crescent achieves fast online presentations with compact proofs, while relying on a two-phase workflow:
\begin{itemize}
  \item Prepare (offline, once per credential): verify issuer signatures, decode and parse the credential into attributes, and produce a Pedersen vector commitment via a Groth16 proof. Reported timings are approximately 27 s for a 2 KB JWT and 140 s for an mDL, with universal-setup parameters of size approximately 661 MB–1.1 GB~\cite[\S4]{cryptoeprint:2024/2013}.
  \item Show (online, per presentation): re-randomize the Groth16 proof and attach a few proofs over committed attributes; typical latencies are approximately 22 ms for JWT and 41.2 ms for mDL, with proofs around 1 KB. Optional device binding increases these costs (e.g., about 315 ms Show, about 184 ms verify; proof size about 15 KB)~\cite[\S4]{cryptoeprint:2024/2013}.
\end{itemize}
Crescent exposes a modular committed-attribute interface that admits sub-provers for range checks, credential linking, and binding to session context or device by proving knowledge of a signature under a committed public key. The principal trade-offs are reliance on pairing-based Groth16 with a trusted setup and large universal parameters, and a comparatively heavy (but amortized) offline Prepare phase; in its current form Crescent is not post-quantum secure~\cite{groth2016size}.

\paragraph{zkID.}
zkID is a modular, transparent (no-CRS) zero-knowledge wrapper for standardized credentials, prioritizing unlinkability and deployability. We adopt the same two-phase workflow as Crescent but avoid a universal setup, and we retain ECDSA compatibility as in Google’s design while moving issuer verification to the offline phase to reduce online verifier work. Key distinctions from prior work include:
\begin{itemize}
  \item Transparency and PQ agility: no trusted setup; standard discrete-log assumptions; an architecture designed to swap in PQ vector commitments in future instantiations.
  \item Deployment compatibility: interoperates with mDL/SD-JWT and existing PKI (ECDSA or RSA) without changes to issuer processes or secure elements.
  \item Reduced online burden: relative to prior ECDSA-based designs, issuer-signature verification and parsing move to the offline Prepare phase, reducing online compute.
  \item Avoidance of pairing-based cryptography: unlike pairing-based schemes, zkID avoids pairing-friendly curves.
  \item Verifier cost: reduces verifier work relative to prior ECDSA-based designs under our constraints while preserving unlinkability under the stated threat model.
\end{itemize}

In summary, zkID shows that a modular, standards-aligned zero-knowledge wrapper over existing credential formats can be built. Compared to pairing-based schemes, it avoids dependence on pairing-friendly curves; compared to universal-setup SNARK wrappers, it avoids a trusted setup; and relative to prior ECDSA-based designs, it reduces verifier cost while preserving unlinkability under our model. The construction is transparent and PQC-agile; the current instantiation is classical and not quantum-resistant.

In our evaluation, zkID achieves practical performance on commodity hardware. For a typical credential with ..., the one-time \emph{Prepare} phase completes in ... seconds on a standard laptop (or mobile) (..., ...GB RAM), while the per-presentation \emph{Show} phase takes less than ... ms on a mid-range smartphone. Proof sizes are approximately ... KB for the full protocol, and verifier time remains below ... ms on a web server. These results demonstrate that zkID meets latency and resource targets for mobile and web deployments, while preserving unlinkability and avoiding trusted setup. Detailed benchmark results and further breakdowns will be provided in Section~\ref{sec:experiments}.

We leave several aspects out of scope for future work, including comprehensive revocation, explicit defenses against hardware side channels, wallet-integration benchmarks, and lattice-based vector commitments for PQ instantiations.